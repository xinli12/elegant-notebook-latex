% Elegant Notebook - Settings File
% This file contains all the package imports and configuration settings

% Load custom metadata
% Elegant Notebook - Metadata
% Customise these values to personalise your notebook

% --- DOCUMENT METADATA ---
% These values will appear on the title page and in headers
\newcommand{\notebooktitle}{Elegant Research Notebook}
\newcommand{\notebooksubtitle}{A Professional \LaTeX{} Template for Academic Writing}
\newcommand{\notebookauthor}{Your Name}
\newcommand{\notebookdate}{\today}
\newcommand{\notebookemail}{your.email@example.com}
\newcommand{\notebookinstitution}{Your Institution}

% Note: PDF metadata properties are automatically set in settings.tex
% after the hyperref package is loaded 

% Load custom colour scheme
% Elegant Notebook - Colour Scheme
% This file defines the colour scheme for the notebook
% Modify these colours to customise the appearance

\usepackage{xcolor}
\usepackage{comment}
\usepackage{etoolbox}

% -----------------------------------------------------
% THEME SELECTION
% -----------------------------------------------------
% The theme can be selected using \usetheme{font-theme}{color-theme}
% Set a default if not already defined
\providecommand{\activetheme}{cambridge-blue}

% -----------------------------------------------------
% THEME DEFINITIONS
% -----------------------------------------------------

% --- COLOUR THEME: CAMBRIDGE BLUE (DEFAULT) ---
\ifx\activetheme\undefined\def\activetheme{cambridge-blue}\fi
\ifdefstring{\activetheme}{cambridge-blue}{%
    % Main theme colours - Official Cambridge Blue (Pantone 557 C)
    \definecolor{maincolor}{RGB}{133,176,154}     % Primary colour (Official Cambridge Blue)
    \definecolor{accentcolor}{RGB}{90,140,115}    % Secondary colour (Darker variant)
    \definecolor{highlightcolor}{RGB}{230,126,34} % Accent colour (Orange) for emphasis

    % Auxiliary colours
    \definecolor{codebg}{RGB}{245,250,247}        % Light background for code
    \definecolor{codefg}{RGB}{45,75,60}           % Darker foreground for code
    \definecolor{softgray}{RGB}{190,210,200}      % Utility grey colour
    \definecolor{titlepagebg}{RGB}{235,245,240}   % Soft background for title page
    \definecolor{inlinecodebg}{RGB}{235,245,240}  % Lighter green-blue background for inline code
    \definecolor{inlinecodefg}{RGB}{0,105,85}     % Slightly darker teal for inline code
}{}

% --- THEME: CAMBRIDGE CLASSIC ---
\ifdefstring{\activetheme}{cambridge-classic}{%
    \definecolor{maincolor}{RGB}{133,176,154}     % Official Cambridge Blue
    \definecolor{accentcolor}{RGB}{64,98,125}     % Cambridge Dark Blue
    \definecolor{highlightcolor}{RGB}{210,180,100}% Cambridge Gold
    \definecolor{codebg}{RGB}{245,248,245}        % Light background for code
    \definecolor{codefg}{RGB}{40,65,80}           % Dark blue-tinted code text
    \definecolor{softgray}{RGB}{180,195,190}      % Blue-green tinted grey
    \definecolor{titlepagebg}{RGB}{235,245,240}   % Soft blue-green background
    \definecolor{inlinecodebg}{RGB}{235,245,242}  % Light blue-green background for inline code
    \definecolor{inlinecodefg}{RGB}{50,90,115}    % Cambridge dark blue for inline code
}{}

% --- THEME: CAMBRIDGE FOREST ---
\ifdefstring{\activetheme}{cambridge-forest}{%
    \definecolor{maincolor}{RGB}{133,176,154}     % Official Cambridge Blue
    \definecolor{accentcolor}{RGB}{65,105,75}     % Deep Forest Green
    \definecolor{highlightcolor}{RGB}{205,170,80} % Autumn Gold
    \definecolor{codebg}{RGB}{245,250,245}        % Light forest-tinted background
    \definecolor{codefg}{RGB}{35,70,45}           % Dark forest green text
    \definecolor{softgray}{RGB}{185,200,185}      % Green-tinted grey
    \definecolor{titlepagebg}{RGB}{240,248,242}   % Soft forest-tinted background
    \definecolor{inlinecodebg}{RGB}{235,245,235}  % Light forest green background for inline code
    \definecolor{inlinecodefg}{RGB}{45,90,55}     % Forest green for inline code
}{}

% --- THEME: CAMBRIDGE SUNSET ---
\ifdefstring{\activetheme}{cambridge-sunset}{%
    \definecolor{maincolor}{RGB}{133,176,154}     % Official Cambridge Blue
    \definecolor{accentcolor}{RGB}{175,95,65}     % Sunset Orange
    \definecolor{highlightcolor}{RGB}{230,180,60} % Golden Sunset
    \definecolor{codebg}{RGB}{250,248,245}        % Warm light background
    \definecolor{codefg}{RGB}{60,60,50}           % Warm dark text
    \definecolor{softgray}{RGB}{200,195,185}      % Warm-tinted grey
    \definecolor{titlepagebg}{RGB}{245,245,240}   % Soft warm background
    \definecolor{inlinecodebg}{RGB}{248,245,240}  % Warm cream background for inline code
    \definecolor{inlinecodefg}{RGB}{160,80,50}    % Sunset orange for inline code
}{}

% --- THEME: CAMBRIDGE RIVER ---
\ifdefstring{\activetheme}{cambridge-river}{%
    \definecolor{maincolor}{RGB}{133,176,154}     % Official Cambridge Blue
    \definecolor{accentcolor}{RGB}{80,130,170}    % River Blue
    \definecolor{highlightcolor}{RGB}{180,210,200}% River Mist
    \definecolor{codebg}{RGB}{240,245,250}        % Cool blue-tinted background
    \definecolor{codefg}{RGB}{40,70,105}          % Deep blue text
    \definecolor{softgray}{RGB}{185,195,205}      % Blue-tinted grey
    \definecolor{titlepagebg}{RGB}{235,242,248}   % Cool light blue background
    \definecolor{inlinecodebg}{RGB}{235,240,248}  % Light blue background for inline code
    \definecolor{inlinecodefg}{RGB}{60,110,150}   % River blue for inline code
}{}

% --- THEME: OXFORD BLUE ---
\ifdefstring{\activetheme}{oxford-blue}{%
    \definecolor{maincolor}{RGB}{0,33,71}        % Oxford Blue
    \definecolor{accentcolor}{RGB}{140,45,25}    % Oxford Red
    \definecolor{highlightcolor}{RGB}{255,205,0} % Gold accent
    \definecolor{codebg}{RGB}{240,240,245}       % Light blue-tinted code background
    \definecolor{codefg}{RGB}{30,30,60}          % Dark blue code foreground
    \definecolor{softgray}{RGB}{190,190,200}     % Blue-tinted grey
    \definecolor{titlepagebg}{RGB}{230,235,245}  % Soft blue background
    \definecolor{inlinecodebg}{RGB}{230,235,245} % Slightly richer blue background for inline code
    \definecolor{inlinecodefg}{RGB}{110,30,20}   % Oxford red for inline code text (using accent colour)
}{}

% --- THEME: OXFORD MIST ---
\ifdefstring{\activetheme}{oxford-mist}{%
    \definecolor{maincolor}{RGB}{0,33,71}         % Oxford Blue
    \definecolor{accentcolor}{RGB}{100,100,130}   % Dusty Blue
    \definecolor{highlightcolor}{RGB}{150,150,170}% Mist Lavender
    \definecolor{codebg}{RGB}{245,245,250}        % Cloudy background
    \definecolor{codefg}{RGB}{40,40,60}           % Slate navy
    \definecolor{softgray}{RGB}{200,200,210}      % Misty grey
    \definecolor{titlepagebg}{RGB}{235,240,245}   % Pale bluish white
    \definecolor{inlinecodebg}{RGB}{235,235,245}  % Richer misty blue for inline code
    \definecolor{inlinecodefg}{RGB}{80,80,120}    % Dusty blue for inline code (from accent)
}{}

% --- THEME: OXFORD MONOCHROME ---
\ifdefstring{\activetheme}{oxford-monochrome}{%
    \definecolor{maincolor}{RGB}{0,33,71}         % Oxford Blue
    \definecolor{accentcolor}{RGB}{50,50,70}      % Charcoal Blue
    \definecolor{highlightcolor}{RGB}{90,90,120}  % Subtle Violet
    \definecolor{codebg}{RGB}{235,235,240}        % Light grey-blue
    \definecolor{codefg}{RGB}{20,30,50}           % Deep navy
    \definecolor{softgray}{RGB}{180,180,190}      % Cool grey
    \definecolor{titlepagebg}{RGB}{225,230,240}   % Very light blue-grey
    \definecolor{inlinecodebg}{RGB}{220,220,235}  % More saturated blue-grey for inline code
    \definecolor{inlinecodefg}{RGB}{40,40,65}     % Deeper charcoal for inline code
}{}

% --- THEME: SUMMER SKIES ---
\ifdefstring{\activetheme}{summer-skies}{%
    \definecolor{maincolor}{RGB}{85,150,205}      % Sky Blue
    \definecolor{accentcolor}{RGB}{55,120,175}    % Deeper Sky Blue
    \definecolor{highlightcolor}{RGB}{240,210,80} % Sunshine Yellow
    \definecolor{codebg}{RGB}{240,248,255}        % Pale sky background
    \definecolor{codefg}{RGB}{30,75,120}          % Deep sky blue text
    \definecolor{softgray}{RGB}{195,210,225}      % Sky-tinted grey
    \definecolor{titlepagebg}{RGB}{235,245,255}   % Soft sky background
    \definecolor{inlinecodebg}{RGB}{230,240,255}  % Light sky blue background for inline code
    \definecolor{inlinecodefg}{RGB}{40,105,160}   % Sky blue for inline code
}{}

% --- THEME: AURORA SKIES ---
\ifdefstring{\activetheme}{aurora-skies}{%
    \definecolor{maincolor}{RGB}{60,125,150}      % Northern Blue
    \definecolor{accentcolor}{RGB}{100,170,120}   % Aurora Green
    \definecolor{highlightcolor}{RGB}{180,140,200}% Aurora Purple
    \definecolor{codebg}{RGB}{240,248,248}        % Cold sky background
    \definecolor{codefg}{RGB}{40,85,100}          % Deep northern blue text
    \definecolor{softgray}{RGB}{185,200,205}      % Northern-tinted grey
    \definecolor{titlepagebg}{RGB}{235,245,245}   % Soft northern sky background
    \definecolor{inlinecodebg}{RGB}{235,245,240}  % Light aurora background for inline code
    \definecolor{inlinecodefg}{RGB}{80,140,100}   % Aurora green for inline code
}{}

% --- THEME: TWILIGHT SKIES ---
\ifdefstring{\activetheme}{twilight-skies}{%
    \definecolor{maincolor}{RGB}{90,100,160}      % Twilight Blue
    \definecolor{accentcolor}{RGB}{150,100,150}   % Twilight Purple
    \definecolor{highlightcolor}{RGB}{230,180,140}% Sunset Gold
    \definecolor{codebg}{RGB}{242,242,250}        % Cool evening background
    \definecolor{codefg}{RGB}{60,60,100}          % Deep twilight text
    \definecolor{softgray}{RGB}{190,190,210}      % Twilight-tinted grey
    \definecolor{titlepagebg}{RGB}{235,235,245}   % Soft twilight background
    \definecolor{inlinecodebg}{RGB}{235,230,245}  % Light twilight background for inline code
    \definecolor{inlinecodefg}{RGB}{130,85,135}   % Twilight purple for inline code
}{}

% --- THEME: MIDNIGHT SKIES ---
\ifdefstring{\activetheme}{midnight-skies}{%
    \definecolor{maincolor}{RGB}{45,60,110}      % Midnight Blue
    \definecolor{accentcolor}{RGB}{90,95,145}    % Softer Midnight Blue
    \definecolor{highlightcolor}{RGB}{180,185,220} % Starlight Blue
    \definecolor{codebg}{RGB}{240,240,248}       % Deep night sky background
    \definecolor{codefg}{RGB}{35,45,85}          % Deep midnight text
    \definecolor{softgray}{RGB}{180,180,200}     % Night-tinted grey
    \definecolor{titlepagebg}{RGB}{232,232,242}  % Soft midnight background
    \definecolor{inlinecodebg}{RGB}{230,230,245} % Light midnight background for inline code
    \definecolor{inlinecodefg}{RGB}{70,80,130}   % Midnight blue for inline code
}{}

% --- THEME: VINTAGE ROSE ---
\ifdefstring{\activetheme}{vintage-rose}{%
    \definecolor{maincolor}{RGB}{196,146,151}     % Dusty Rose (Main colour)
    \definecolor{accentcolor}{RGB}{156,110,115}   % Muted Berry (Secondary colour)
    \definecolor{highlightcolor}{RGB}{242,206,184}% Pale Peach (Highlight)
    \definecolor{codebg}{RGB}{253,248,249}        % Light Pinkish Cream (Code background)
    \definecolor{codefg}{RGB}{105,73,77}          % Brownish Purple (Code foreground)
    \definecolor{softgray}{RGB}{207,192,194}      % Rose-tinted Grey (Utility grey)
    \definecolor{titlepagebg}{RGB}{249,240,241}   % Soft Rose White (Title page background)
    \definecolor{inlinecodebg}{RGB}{248,237,240}  % Deeper rose tint for inline code
    \definecolor{inlinecodefg}{RGB}{140,90,95}    % Richer berry tone for inline code
}{}

% --- THEME: EMERALD ---
\ifdefstring{\activetheme}{emerald}{%
    \definecolor{maincolor}{RGB}{0,112,74}        % Emerald Green
    \definecolor{accentcolor}{RGB}{52,86,125}     % Ocean Blue
    \definecolor{highlightcolor}{RGB}{237,169,33} % Golden Yellow
    \definecolor{codebg}{RGB}{240,248,245}        % Light green-tinted code background
    \definecolor{codefg}{RGB}{20,60,40}           % Dark green code foreground
    \definecolor{softgray}{RGB}{180,200,190}      % Green-tinted grey
    \definecolor{titlepagebg}{RGB}{230,245,240}   % Soft green background
    \definecolor{inlinecodebg}{RGB}{232,245,238}  % Richer emerald tint for inline code
    \definecolor{inlinecodefg}{RGB}{0,90,60}      % Deep emerald for inline code text
}{}

% --- THEME: ROYAL PURPLE ---
\ifdefstring{\activetheme}{royal-purple}{%
    \definecolor{maincolor}{RGB}{76,40,130}       % Royal Purple
    \definecolor{accentcolor}{RGB}{212,175,55}    % Gold
    \definecolor{highlightcolor}{RGB}{220,20,60}  % Crimson
    \definecolor{codebg}{RGB}{245,240,255}        % Light purple background
    \definecolor{codefg}{RGB}{50,30,80}           % Dark purple foreground
    \definecolor{softgray}{RGB}{200,190,210}      % Purple-tinted grey
    \definecolor{titlepagebg}{RGB}{240,235,250}   % Soft purple background
    \definecolor{inlinecodebg}{RGB}{235,230,250}  % Richer purple for inline code
    \definecolor{inlinecodefg}{RGB}{90,50,150}    % Deeper royal purple for inline code
}{}

% --- THEME: ARCTIC ---
\ifdefstring{\activetheme}{arctic}{%
    \definecolor{maincolor}{RGB}{64,103,129}      % Arctic Blue
    \definecolor{accentcolor}{RGB}{25,64,86}      % Deep Arctic
    \definecolor{highlightcolor}{RGB}{180,210,220}% Ice Blue
    \definecolor{codebg}{RGB}{235,245,250}        % Snow White
    \definecolor{codefg}{RGB}{40,60,80}           % Midnight Blue
    \definecolor{softgray}{RGB}{190,205,215}      % Glacier Grey
    \definecolor{titlepagebg}{RGB}{230,240,245}   % Frost White
    \definecolor{inlinecodebg}{RGB}{225,238,248}  % Deeper ice blue for inline code
    \definecolor{inlinecodefg}{RGB}{25,80,105}    % Richer arctic blue for inline code
}{}

% --- THEME: AUTUMN ---
\ifdefstring{\activetheme}{autumn}{%
    \definecolor{maincolor}{RGB}{165,82,42}       % Sienna Brown
    \definecolor{accentcolor}{RGB}{205,133,63}    % Peru
    \definecolor{highlightcolor}{RGB}{220,105,30} % Burnt Orange
    \definecolor{codebg}{RGB}{252,248,240}        % Cream background
    \definecolor{codefg}{RGB}{80,40,20}           % Dark Brown foreground
    \definecolor{softgray}{RGB}{210,200,180}      % Beige Grey
    \definecolor{titlepagebg}{RGB}{250,245,235}   % Pale Cream
    \definecolor{inlinecodebg}{RGB}{248,242,233}  % Richer autumn gold for inline code
    \definecolor{inlinecodefg}{RGB}{170,90,45}    % Deeper autumn tone for inline code
}{}

% -----------------------------------------------------
% CUSTOM THEME 
% -----------------------------------------------------
% Create your own theme by adding a new selection block:
%
% \ifdefstring{\activetheme}{your-theme-name}{%
%     \definecolor{maincolor}{RGB}{0,0,0}           % Primary colour
%     \definecolor{accentcolor}{RGB}{0,0,0}         % Secondary colour
%     \definecolor{highlightcolor}{RGB}{0,0,0}      % Accent colour for emphasis
%     \definecolor{codebg}{RGB}{0,0,0}              % Background for code blocks
%     \definecolor{codefg}{RGB}{0,0,0}              % Text for code blocks
%     \definecolor{softgray}{RGB}{0,0,0}            % Utility grey colour
%     \definecolor{titlepagebg}{RGB}{0,0,0}         % Background for title page
%     \definecolor{inlinecodebg}{RGB}{0,0,0}        % Background for inline code
%     \definecolor{inlinecodefg}{RGB}{0,0,0}        % Text colour for inline code
% }{}

% Setting up geometry for balanced margins
\usepackage[margin=1in, headheight=14pt]{geometry} % Increased headheight to prevent fancyhdr warnings

% Including essential packages for language, encoding, and functionality
\usepackage[utf8]{inputenc}
\usepackage[T1]{fontenc}
\usepackage[british]{babel} % British English spelling and conventions
\usepackage{csquotes} % Recommended when using babel with biblatex

% Adding mathematical packages first to avoid conflicts
\usepackage{amsmath}
\usepackage{amssymb}

% --- FONT SELECTION ---
% The font theme is now set using the \usetheme command in elegant-notebook.tex
% \fonttheme is defined by the theme system, so we just need to use it
\usepackage{microtype} % Improved typography for all fonts
\usepackage[varqu,varl]{inconsolata} % Better monospace base font
\usepackage{etoolbox} % Required for string comparison

% Save definition of \Bbbk from amssymb to avoid redefinition errors
\let\amssymbBbbk\Bbbk

% Modern theme (default)
\ifx\fonttheme\@empty\def\fonttheme{modern}\fi
\ifdefstring{\fonttheme}{modern}{%
    \usepackage{charter} % Serif
    \usepackage[scaled=0.85]{berasans} % Sans-serif
    \usepackage{newtxmath} % Compatible mathematics font
}{} % End modern theme

% Restore original \Bbbk definition
\let\Bbbk\amssymbBbbk

% Classic theme
\ifdefstring{\fonttheme}{classic}{%
    \usepackage{palatino} % Palatino for serif
    \usepackage[scale=0.9]{tgheros} % TeX Gyre Heros for sans-serif
    \usepackage{newpxmath} % Palatino-compatible mathematics
}{} % End classic theme

% Elegant theme
\ifdefstring{\fonttheme}{elegant}{%
    \usepackage{ebgaramond} % Garamond for elegant serif
    \usepackage[scale=0.85]{sourcesanspro} % Source Sans Pro
    \usepackage{newpxmath} % Compatible math font
}{} % End elegant theme

% Academic theme
\ifdefstring{\fonttheme}{academic}{%
    \usepackage{libertinus} % Libertinus family (serif, sans, and maths)
    \usepackage{libertinust1math} % Matching maths font
}{} % End academic theme

% Scientific theme
\ifdefstring{\fonttheme}{scientific}{%
    \usepackage{stix2} % STIX Two for scientific documents
    \usepackage[scale=0.9]{cabin} % Cabin for sans-serif
    % STIX2 already includes mathematics support
}{} % End scientific theme

% Professional theme
\ifdefstring{\fonttheme}{professional}{%
    \usepackage{newpxtext} % Palatino clone with enhanced features
    \usepackage[scale=0.92]{sourcesanspro} % Source Sans Pro for headings
    \usepackage{newpxmath} % Matching mathematics
}{} % End professional theme

% Contemporary theme
\ifdefstring{\fonttheme}{contemporary}{%
    \usepackage[sfdefault,scaled=.85]{FiraSans} % Modern sans-serif
    \usepackage{newtxsf} % Matching sans-serif maths
}{} % End contemporary theme

% --- FONT SHOWCASE COMMANDS ---
% These commands are for showcasing the font themes in the document.
\newcommand{\showcasemodern}[1]{{\fontfamily{bch}\selectfont #1}}
\newcommand{\showcaseclassic}[1]{{\fontfamily{ppl}\selectfont #1}}
\newcommand{\showcaseelegant}[1]{{\fontfamily{EBGaramond-LF}\selectfont #1}}
\newcommand{\showcaseacademic}[1]{{\fontfamily{LibertinusT1}\selectfont #1}}
\newcommand{\showcasescientific}[1]{{\fontfamily{STIXTwoText-TLF}\selectfont #1}}
\newcommand{\showcaseprofessional}[1]{{\fontfamily{pxr}\selectfont #1}}
\newcommand{\showcasecontemporary}[1]{{\fontfamily{FiraSans-TLF}\selectfont #1}}

% Setting up hyperref for clickable links with subtle styling
\usepackage{hyperref}
\hypersetup{
    colorlinks=true,
    linkcolor=accentcolor, % Use accent colour for links
    urlcolor=accentcolor,
    citecolor=accentcolor,
    pdfborderstyle={/S/U/W 0.5}, % Thinner underline
    % PDF metadata properties
    pdftitle={\notebooktitle},
    pdfauthor={\notebookauthor},
    pdfsubject={\notebooksubtitle},
    pdfcreator={LaTeX with Elegant Notebook Template},
    pdfproducer={LaTeX}
}

% Enhancing lists with custom spacing and symbols
\usepackage{enumitem}
\setlist[itemize]{leftmargin=*, itemsep=0.5ex, topsep=0.5ex, label=\textcolor{accentcolor}{\textbullet}} % Custom bullet
\setlist[enumerate]{leftmargin=*, itemsep=0.5ex, topsep=0.5ex, label=\textcolor{accentcolor}{\arabic*.}}

% Adding line spacing for readability
\usepackage{setspace}
\setstretch{1.15} % Slightly reduced for a denser but still readable look

% Including graphics, table, and caption support
\usepackage{graphicx}
\usepackage{booktabs} % For professional tables
\usepackage{caption}
\captionsetup{font={small,sf}, labelfont={bf,color=accentcolor}, skip=7pt, singlelinecheck=false, format=hang}

% Configuring inline code and environments with tcolorbox
\usepackage[many]{tcolorbox}
\tcbuselibrary{skins,breakable,hooks} % Load tcolorbox libraries for more features

% Improved inline code style
\newtcbox{\inlinecode}{
    nobeforeafter,
    colback=inlinecodebg,
    colframe=inlinecodefg!50,
    boxrule=0.2pt,
    arc=1.0mm,
    left=2pt, right=2pt, top=1.5pt, bottom=1.5pt,
    boxsep=0.7pt,
    on line,
    fontupper=\small\ttfamily\color{inlinecodefg},
    enlarge left by=0mm,
    enlarge right by=0mm
}

% --- ENHANCED TCOLORBOX STYLES ---
\newtcolorbox{notetbox}[1][Note]{
    enhanced, % Use enhanced features
    colback=maincolor!10,
    colframe=maincolor,
    fonttitle=\sffamily\bfseries,
    coltitle=white, % White text on coloured background
    title=#1,
    left=2mm, right=2mm, top=2mm, bottom=2mm,
    arc=1mm,
    breakable % Allow box to break across pages
}

\newtcolorbox{exampletbox}[1][Example]{
    enhanced,
    colback=accentcolor!10,
    colframe=accentcolor,
    fonttitle=\sffamily\bfseries,
    coltitle=white,
    title=#1,
    left=2mm, right=2mm, top=2mm, bottom=2mm,
    arc=2mm,
    breakable
}

% Adding support for code listings
\usepackage{listings}
\lstset{
    basicstyle=\ttfamily\small\color{codefg},
    backgroundcolor=\color{codebg},
    frame=single, % Complete border
    framerule=0.5pt,
    rulecolor=\color{maincolor!50},
    breaklines=true,
    numbers=left,
    numberstyle=\tiny\color{maincolor!70!black},
    keywordstyle=\color{accentcolor}\bfseries,
    commentstyle=\color{maincolor!60!black}\itshape,
    stringstyle=\color{highlightcolor!90!black}, % Style for strings
    identifierstyle=\color{codefg}, % Style for identifiers
    tabsize=2,
    showspaces=false,
    showstringspaces=false,
    captionpos=b, % Position of the caption (bottom)
    aboveskip=1.5em,
    belowskip=1.5em,
    extendedchars=true,
    xleftmargin=18pt, % Increase left margin to ensure enough space for line numbers
    xrightmargin=8pt, % Increase right margin
    numbersep=8pt, % Increase space between line numbers and code
    framexleftmargin=15pt, % Ensure appropriate left frame margin
    framexrightmargin=5pt, % Right frame margin
    framextopmargin=6pt, % Top internal padding
    framexbottommargin=6pt, % Bottom internal padding
    resetmargins=true, % Reset margins
    literate=% Add specific character renderings if needed
        {á}{{\'a}}1 {é}{{\'e}}1 {í}{{\'i}}1 {ó}{{\'o}}1 {ú}{{\'u}}1
        {Á}{{\'A}}1 {É}{{\'E}}1 {Í}{{\'I}}1 {Ó}{{\'O}}1 {Ú}{{\'U}}1
        {ñ}{{\~n}}1 {Ñ}{{\~N}}1
}

% Adding support for simple diagrams with TikZ
\usepackage{tikz}
\usetikzlibrary{shapes.geometric, arrows.meta, positioning, shadows, patterns} % More TikZ libraries

% --- ENHANCED SECTION STYLING ---
\usepackage{titlesec}

% Chapter styling
\titleformat{\chapter}[display]
    {\sffamily\huge\bfseries\color{maincolor}}
    {\textcolor{accentcolor}{\chaptertitlename\ \thechapter}}{1ex}
    {\vspace{0.5ex}\color{maincolor}}
    [\vspace{0.5ex}{\color{maincolor!40}\titlerule[1.5pt]}]

\titleformat{\section}
    {\sffamily\Large\bfseries\color{maincolor}}
    {\thesection}{0.8em}{}
    [\color{maincolor!30}\titlerule] % Adds a rule below the title
\titleformat{\subsection}
    {\sffamily\large\bfseries\color{accentcolor}} % Use accent colour
    {\thesubsection}{0.8em}{}
\titleformat{\subsubsection}
    {\sffamily\normalsize\bfseries\color{accentcolor!80!black}}
    {\thesubsubsection}{0.8em}{}

\titlespacing*{\chapter}{0pt}{0pt}{3ex plus 1ex}
\titlespacing*{\section}{0pt}{2.5ex plus 1ex minus .2ex}{1.5ex plus .2ex}
\titlespacing*{\subsection}{0pt}{2ex plus .8ex minus .2ex}{1ex plus .2ex}
\titlespacing*{\subsubsection}{0pt}{1.5ex plus .6ex minus .2ex}{0.8ex plus .2ex}

% Setting up bibliography with biblatex
\usepackage[backend=biber, style=numeric, sorting=none]{biblatex}
\addbibresource{references.bib} % Reference file

% Creating a custom header and footer
\usepackage{fancyhdr}
\pagestyle{fancy}
\fancyhf{}
\fancyhead[L]{\sffamily\small\textcolor{maincolor!90!black}{\notebooktitle}}
\fancyhead[R]{\sffamily\small\textcolor{maincolor!90!black}{\nouppercase{\leftmark}}} % Section in header
\fancyfoot[C]{\sffamily\small\textcolor{maincolor!90!black}{\thepage}}
\renewcommand{\headrulewidth}{0.5pt}
\renewcommand{\footrulewidth}{0pt} % Remove footer rule for a cleaner look
\renewcommand{\headrule}{\color{maincolor!70}\hrule} 