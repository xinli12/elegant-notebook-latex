% Elegant Notebook Template
% A professional LaTeX template with customisable settings for academic and research work

\documentclass[a4paper,11pt]{report}

% Load theme switching functionality
% Elegant Notebook - Theme Switcher
% This file provides a one-step solution for changing both color and font themes

% Define the theme switcher command
\newcommand{\usetheme}[2]{%
    % Set font theme
    \def\fonttheme{#1}%
    
    % Set color theme
    \def\activetheme{#2}%
}

% Default theme if not specified elsewhere
\providecommand{\fonttheme}{modern}
\providecommand{\activetheme}{cambridge-blue} 

% Select your theme - first parameter is font theme, second is colour theme
% Font themes: modern, classic, elegant, academic, scientific, professional, contemporary
% Colour themes: cambridge-blue, cambridge-classic, cambridge-forest, cambridge-sunset, cambridge-river,
%              oxford-blue, oxford-mist, oxford-monochrome, summer-skies, aurora-skies, 
%              twilight-skies, midnight-skies, vintage-rose, emerald, royal-purple, arctic, autumn
\usetheme{modern}{cambridge-blue}

% Load the Elegant Notebook settings
% The font theme is set by the \usetheme command above
% Elegant Notebook - Settings File
% This file contains all the package imports and configuration settings

% Load custom metadata
% Elegant Notebook - Metadata
% This is where you can add your personal touch!
% Customise these values to make the notebook truly yours.

% --- DOCUMENT METADATA ---
% These details will appear on your title page and in the page headers.
\newcommand{\notebooktitle}{My Awesome Research Notebook}
\newcommand{\notebooksubtitle}{A Subtitle for My Brilliant Work}
\newcommand{\notebookauthor}{Your Name}
\newcommand{\notebookdate}{\today}
\newcommand{\notebookemail}{your.email@example.com}
\newcommand{\notebookinstitution}{Your Institution}

% Note: The PDF metadata properties (like title and author) are automatically
% set in config/settings.tex using these values. No need to change anything there! 

% Load custom colour scheme
% Elegant Notebook - Colour Scheme
% This file defines the colour scheme for the notebook
% Modify these colours to customise the appearance

\usepackage{xcolor}
\usepackage{comment}
\usepackage{etoolbox}

% -----------------------------------------------------
% THEME SELECTION
% -----------------------------------------------------
% The theme can be selected using \usetheme{font-theme}{color-theme}
% Set a default if not already defined
\providecommand{\activetheme}{cambridge-blue}

% -----------------------------------------------------
% THEME DEFINITIONS
% -----------------------------------------------------

% --- COLOUR THEME: CAMBRIDGE BLUE (DEFAULT) ---
\ifx\activetheme\undefined\def\activetheme{cambridge-blue}\fi
\ifdefstring{\activetheme}{cambridge-blue}{%
    % Main theme colours - Official Cambridge Blue (Pantone 557 C)
    \definecolor{maincolor}{RGB}{133,176,154}     % Primary colour (Official Cambridge Blue)
    \definecolor{accentcolor}{RGB}{90,140,115}    % Secondary colour (Darker variant)
    \definecolor{highlightcolor}{RGB}{230,126,34} % Accent colour (Orange) for emphasis

    % Auxiliary colours
    \definecolor{codebg}{RGB}{245,250,247}        % Light background for code
    \definecolor{codefg}{RGB}{45,75,60}           % Darker foreground for code
    \definecolor{softgray}{RGB}{190,210,200}      % Utility grey colour
    \definecolor{titlepagebg}{RGB}{235,245,240}   % Soft background for title page
    \definecolor{inlinecodebg}{RGB}{235,245,240}  % Lighter green-blue background for inline code
    \definecolor{inlinecodefg}{RGB}{0,105,85}     % Slightly darker teal for inline code
}{}

% --- THEME: CAMBRIDGE CLASSIC ---
\ifdefstring{\activetheme}{cambridge-classic}{%
    \definecolor{maincolor}{RGB}{133,176,154}     % Official Cambridge Blue
    \definecolor{accentcolor}{RGB}{64,98,125}     % Cambridge Dark Blue
    \definecolor{highlightcolor}{RGB}{210,180,100}% Cambridge Gold
    \definecolor{codebg}{RGB}{245,248,245}        % Light background for code
    \definecolor{codefg}{RGB}{40,65,80}           % Dark blue-tinted code text
    \definecolor{softgray}{RGB}{180,195,190}      % Blue-green tinted grey
    \definecolor{titlepagebg}{RGB}{235,245,240}   % Soft blue-green background
    \definecolor{inlinecodebg}{RGB}{235,245,242}  % Light blue-green background for inline code
    \definecolor{inlinecodefg}{RGB}{50,90,115}    % Cambridge dark blue for inline code
}{}

% --- THEME: CAMBRIDGE FOREST ---
\ifdefstring{\activetheme}{cambridge-forest}{%
    \definecolor{maincolor}{RGB}{133,176,154}     % Official Cambridge Blue
    \definecolor{accentcolor}{RGB}{65,105,75}     % Deep Forest Green
    \definecolor{highlightcolor}{RGB}{205,170,80} % Autumn Gold
    \definecolor{codebg}{RGB}{245,250,245}        % Light forest-tinted background
    \definecolor{codefg}{RGB}{35,70,45}           % Dark forest green text
    \definecolor{softgray}{RGB}{185,200,185}      % Green-tinted grey
    \definecolor{titlepagebg}{RGB}{240,248,242}   % Soft forest-tinted background
    \definecolor{inlinecodebg}{RGB}{235,245,235}  % Light forest green background for inline code
    \definecolor{inlinecodefg}{RGB}{45,90,55}     % Forest green for inline code
}{}

% --- THEME: CAMBRIDGE SUNSET ---
\ifdefstring{\activetheme}{cambridge-sunset}{%
    \definecolor{maincolor}{RGB}{133,176,154}     % Official Cambridge Blue
    \definecolor{accentcolor}{RGB}{175,95,65}     % Sunset Orange
    \definecolor{highlightcolor}{RGB}{230,180,60} % Golden Sunset
    \definecolor{codebg}{RGB}{250,248,245}        % Warm light background
    \definecolor{codefg}{RGB}{60,60,50}           % Warm dark text
    \definecolor{softgray}{RGB}{200,195,185}      % Warm-tinted grey
    \definecolor{titlepagebg}{RGB}{245,245,240}   % Soft warm background
    \definecolor{inlinecodebg}{RGB}{248,245,240}  % Warm cream background for inline code
    \definecolor{inlinecodefg}{RGB}{160,80,50}    % Sunset orange for inline code
}{}

% --- THEME: CAMBRIDGE RIVER ---
\ifdefstring{\activetheme}{cambridge-river}{%
    \definecolor{maincolor}{RGB}{133,176,154}     % Official Cambridge Blue
    \definecolor{accentcolor}{RGB}{80,130,170}    % River Blue
    \definecolor{highlightcolor}{RGB}{180,210,200}% River Mist
    \definecolor{codebg}{RGB}{240,245,250}        % Cool blue-tinted background
    \definecolor{codefg}{RGB}{40,70,105}          % Deep blue text
    \definecolor{softgray}{RGB}{185,195,205}      % Blue-tinted grey
    \definecolor{titlepagebg}{RGB}{235,242,248}   % Cool light blue background
    \definecolor{inlinecodebg}{RGB}{235,240,248}  % Light blue background for inline code
    \definecolor{inlinecodefg}{RGB}{60,110,150}   % River blue for inline code
}{}

% --- THEME: OXFORD BLUE ---
\ifdefstring{\activetheme}{oxford-blue}{%
    \definecolor{maincolor}{RGB}{0,33,71}        % Oxford Blue
    \definecolor{accentcolor}{RGB}{140,45,25}    % Oxford Red
    \definecolor{highlightcolor}{RGB}{255,205,0} % Gold accent
    \definecolor{codebg}{RGB}{240,240,245}       % Light blue-tinted code background
    \definecolor{codefg}{RGB}{30,30,60}          % Dark blue code foreground
    \definecolor{softgray}{RGB}{190,190,200}     % Blue-tinted grey
    \definecolor{titlepagebg}{RGB}{230,235,245}  % Soft blue background
    \definecolor{inlinecodebg}{RGB}{230,235,245} % Slightly richer blue background for inline code
    \definecolor{inlinecodefg}{RGB}{110,30,20}   % Oxford red for inline code text (using accent colour)
}{}

% --- THEME: OXFORD MIST ---
\ifdefstring{\activetheme}{oxford-mist}{%
    \definecolor{maincolor}{RGB}{0,33,71}         % Oxford Blue
    \definecolor{accentcolor}{RGB}{100,100,130}   % Dusty Blue
    \definecolor{highlightcolor}{RGB}{150,150,170}% Mist Lavender
    \definecolor{codebg}{RGB}{245,245,250}        % Cloudy background
    \definecolor{codefg}{RGB}{40,40,60}           % Slate navy
    \definecolor{softgray}{RGB}{200,200,210}      % Misty grey
    \definecolor{titlepagebg}{RGB}{235,240,245}   % Pale bluish white
    \definecolor{inlinecodebg}{RGB}{235,235,245}  % Richer misty blue for inline code
    \definecolor{inlinecodefg}{RGB}{80,80,120}    % Dusty blue for inline code (from accent)
}{}

% --- THEME: OXFORD MONOCHROME ---
\ifdefstring{\activetheme}{oxford-monochrome}{%
    \definecolor{maincolor}{RGB}{0,33,71}         % Oxford Blue
    \definecolor{accentcolor}{RGB}{50,50,70}      % Charcoal Blue
    \definecolor{highlightcolor}{RGB}{90,90,120}  % Subtle Violet
    \definecolor{codebg}{RGB}{235,235,240}        % Light grey-blue
    \definecolor{codefg}{RGB}{20,30,50}           % Deep navy
    \definecolor{softgray}{RGB}{180,180,190}      % Cool grey
    \definecolor{titlepagebg}{RGB}{225,230,240}   % Very light blue-grey
    \definecolor{inlinecodebg}{RGB}{220,220,235}  % More saturated blue-grey for inline code
    \definecolor{inlinecodefg}{RGB}{40,40,65}     % Deeper charcoal for inline code
}{}

% --- THEME: SUMMER SKIES ---
\ifdefstring{\activetheme}{summer-skies}{%
    \definecolor{maincolor}{RGB}{85,150,205}      % Sky Blue
    \definecolor{accentcolor}{RGB}{55,120,175}    % Deeper Sky Blue
    \definecolor{highlightcolor}{RGB}{240,210,80} % Sunshine Yellow
    \definecolor{codebg}{RGB}{240,248,255}        % Pale sky background
    \definecolor{codefg}{RGB}{30,75,120}          % Deep sky blue text
    \definecolor{softgray}{RGB}{195,210,225}      % Sky-tinted grey
    \definecolor{titlepagebg}{RGB}{235,245,255}   % Soft sky background
    \definecolor{inlinecodebg}{RGB}{230,240,255}  % Light sky blue background for inline code
    \definecolor{inlinecodefg}{RGB}{40,105,160}   % Sky blue for inline code
}{}

% --- THEME: AURORA SKIES ---
\ifdefstring{\activetheme}{aurora-skies}{%
    \definecolor{maincolor}{RGB}{60,125,150}      % Northern Blue
    \definecolor{accentcolor}{RGB}{100,170,120}   % Aurora Green
    \definecolor{highlightcolor}{RGB}{180,140,200}% Aurora Purple
    \definecolor{codebg}{RGB}{240,248,248}        % Cold sky background
    \definecolor{codefg}{RGB}{40,85,100}          % Deep northern blue text
    \definecolor{softgray}{RGB}{185,200,205}      % Northern-tinted grey
    \definecolor{titlepagebg}{RGB}{235,245,245}   % Soft northern sky background
    \definecolor{inlinecodebg}{RGB}{235,245,240}  % Light aurora background for inline code
    \definecolor{inlinecodefg}{RGB}{80,140,100}   % Aurora green for inline code
}{}

% --- THEME: TWILIGHT SKIES ---
\ifdefstring{\activetheme}{twilight-skies}{%
    \definecolor{maincolor}{RGB}{90,100,160}      % Twilight Blue
    \definecolor{accentcolor}{RGB}{150,100,150}   % Twilight Purple
    \definecolor{highlightcolor}{RGB}{230,180,140}% Sunset Gold
    \definecolor{codebg}{RGB}{242,242,250}        % Cool evening background
    \definecolor{codefg}{RGB}{60,60,100}          % Deep twilight text
    \definecolor{softgray}{RGB}{190,190,210}      % Twilight-tinted grey
    \definecolor{titlepagebg}{RGB}{235,235,245}   % Soft twilight background
    \definecolor{inlinecodebg}{RGB}{235,230,245}  % Light twilight background for inline code
    \definecolor{inlinecodefg}{RGB}{130,85,135}   % Twilight purple for inline code
}{}

% --- THEME: MIDNIGHT SKIES ---
\ifdefstring{\activetheme}{midnight-skies}{%
    \definecolor{maincolor}{RGB}{45,60,110}      % Midnight Blue
    \definecolor{accentcolor}{RGB}{90,95,145}    % Softer Midnight Blue
    \definecolor{highlightcolor}{RGB}{180,185,220} % Starlight Blue
    \definecolor{codebg}{RGB}{240,240,248}       % Deep night sky background
    \definecolor{codefg}{RGB}{35,45,85}          % Deep midnight text
    \definecolor{softgray}{RGB}{180,180,200}     % Night-tinted grey
    \definecolor{titlepagebg}{RGB}{232,232,242}  % Soft midnight background
    \definecolor{inlinecodebg}{RGB}{230,230,245} % Light midnight background for inline code
    \definecolor{inlinecodefg}{RGB}{70,80,130}   % Midnight blue for inline code
}{}

% --- THEME: VINTAGE ROSE ---
\ifdefstring{\activetheme}{vintage-rose}{%
    \definecolor{maincolor}{RGB}{196,146,151}     % Dusty Rose (Main colour)
    \definecolor{accentcolor}{RGB}{156,110,115}   % Muted Berry (Secondary colour)
    \definecolor{highlightcolor}{RGB}{242,206,184}% Pale Peach (Highlight)
    \definecolor{codebg}{RGB}{253,248,249}        % Light Pinkish Cream (Code background)
    \definecolor{codefg}{RGB}{105,73,77}          % Brownish Purple (Code foreground)
    \definecolor{softgray}{RGB}{207,192,194}      % Rose-tinted Grey (Utility grey)
    \definecolor{titlepagebg}{RGB}{249,240,241}   % Soft Rose White (Title page background)
    \definecolor{inlinecodebg}{RGB}{248,237,240}  % Deeper rose tint for inline code
    \definecolor{inlinecodefg}{RGB}{140,90,95}    % Richer berry tone for inline code
}{}

% --- THEME: EMERALD ---
\ifdefstring{\activetheme}{emerald}{%
    \definecolor{maincolor}{RGB}{0,112,74}        % Emerald Green
    \definecolor{accentcolor}{RGB}{52,86,125}     % Ocean Blue
    \definecolor{highlightcolor}{RGB}{237,169,33} % Golden Yellow
    \definecolor{codebg}{RGB}{240,248,245}        % Light green-tinted code background
    \definecolor{codefg}{RGB}{20,60,40}           % Dark green code foreground
    \definecolor{softgray}{RGB}{180,200,190}      % Green-tinted grey
    \definecolor{titlepagebg}{RGB}{230,245,240}   % Soft green background
    \definecolor{inlinecodebg}{RGB}{232,245,238}  % Richer emerald tint for inline code
    \definecolor{inlinecodefg}{RGB}{0,90,60}      % Deep emerald for inline code text
}{}

% --- THEME: ROYAL PURPLE ---
\ifdefstring{\activetheme}{royal-purple}{%
    \definecolor{maincolor}{RGB}{76,40,130}       % Royal Purple
    \definecolor{accentcolor}{RGB}{212,175,55}    % Gold
    \definecolor{highlightcolor}{RGB}{220,20,60}  % Crimson
    \definecolor{codebg}{RGB}{245,240,255}        % Light purple background
    \definecolor{codefg}{RGB}{50,30,80}           % Dark purple foreground
    \definecolor{softgray}{RGB}{200,190,210}      % Purple-tinted grey
    \definecolor{titlepagebg}{RGB}{240,235,250}   % Soft purple background
    \definecolor{inlinecodebg}{RGB}{235,230,250}  % Richer purple for inline code
    \definecolor{inlinecodefg}{RGB}{90,50,150}    % Deeper royal purple for inline code
}{}

% --- THEME: ARCTIC ---
\ifdefstring{\activetheme}{arctic}{%
    \definecolor{maincolor}{RGB}{64,103,129}      % Arctic Blue
    \definecolor{accentcolor}{RGB}{25,64,86}      % Deep Arctic
    \definecolor{highlightcolor}{RGB}{180,210,220}% Ice Blue
    \definecolor{codebg}{RGB}{235,245,250}        % Snow White
    \definecolor{codefg}{RGB}{40,60,80}           % Midnight Blue
    \definecolor{softgray}{RGB}{190,205,215}      % Glacier Grey
    \definecolor{titlepagebg}{RGB}{230,240,245}   % Frost White
    \definecolor{inlinecodebg}{RGB}{225,238,248}  % Deeper ice blue for inline code
    \definecolor{inlinecodefg}{RGB}{25,80,105}    % Richer arctic blue for inline code
}{}

% --- THEME: AUTUMN ---
\ifdefstring{\activetheme}{autumn}{%
    \definecolor{maincolor}{RGB}{165,82,42}       % Sienna Brown
    \definecolor{accentcolor}{RGB}{205,133,63}    % Peru
    \definecolor{highlightcolor}{RGB}{220,105,30} % Burnt Orange
    \definecolor{codebg}{RGB}{252,248,240}        % Cream background
    \definecolor{codefg}{RGB}{80,40,20}           % Dark Brown foreground
    \definecolor{softgray}{RGB}{210,200,180}      % Beige Grey
    \definecolor{titlepagebg}{RGB}{250,245,235}   % Pale Cream
    \definecolor{inlinecodebg}{RGB}{248,242,233}  % Richer autumn gold for inline code
    \definecolor{inlinecodefg}{RGB}{170,90,45}    % Deeper autumn tone for inline code
}{}

% -----------------------------------------------------
% CUSTOM THEME 
% -----------------------------------------------------
% Create your own theme by adding a new selection block:
%
% \ifdefstring{\activetheme}{your-theme-name}{%
%     \definecolor{maincolor}{RGB}{0,0,0}           % Primary colour
%     \definecolor{accentcolor}{RGB}{0,0,0}         % Secondary colour
%     \definecolor{highlightcolor}{RGB}{0,0,0}      % Accent colour for emphasis
%     \definecolor{codebg}{RGB}{0,0,0}              % Background for code blocks
%     \definecolor{codefg}{RGB}{0,0,0}              % Text for code blocks
%     \definecolor{softgray}{RGB}{0,0,0}            % Utility grey colour
%     \definecolor{titlepagebg}{RGB}{0,0,0}         % Background for title page
%     \definecolor{inlinecodebg}{RGB}{0,0,0}        % Background for inline code
%     \definecolor{inlinecodefg}{RGB}{0,0,0}        % Text colour for inline code
% }{}

% Setting up geometry for balanced margins
\usepackage[margin=0.8in, headheight=14pt]{geometry} % Increased headheight to prevent fancyhdr warnings

% Including essential packages for language, encoding, and functionality
\usepackage[utf8]{inputenc}
\usepackage[T1]{fontenc}
\usepackage[british]{babel} % British English spelling and conventions
\usepackage{csquotes} % Recommended when using babel with biblatex

% Adding mathematical packages first to avoid conflicts
\usepackage{amsmath}
\usepackage{amssymb}

% --- FONT SELECTION ---
% The font theme is now set using the \usetheme command in elegant-notebook.tex
% \fonttheme is defined by the theme system, so we just need to use it
\usepackage{microtype} % Improved typography for all fonts
\usepackage[varqu,varl]{inconsolata} % Better monospace base font
\usepackage{etoolbox} % Required for string comparison

% Save definition of \Bbbk from amssymb to avoid redefinition errors
\let\amssymbBbbk\Bbbk

% Modern theme (default)
\ifx\fonttheme\@empty\def\fonttheme{modern}\fi
\ifdefstring{\fonttheme}{modern}{%
    \usepackage{charter} % Serif
    \usepackage[scaled=0.85]{berasans} % Sans-serif
    \usepackage{newtxmath} % Compatible mathematics font
}{} % End modern theme

% Restore original \Bbbk definition
\let\Bbbk\amssymbBbbk

% Classic theme
\ifdefstring{\fonttheme}{classic}{%
    \usepackage{palatino} % Palatino for serif
    \usepackage[scale=0.9]{tgheros} % TeX Gyre Heros for sans-serif
    \usepackage{newpxmath} % Palatino-compatible mathematics
}{} % End classic theme

% Elegant theme
\ifdefstring{\fonttheme}{elegant}{%
    \usepackage{ebgaramond} % Garamond for elegant serif
    \usepackage[scale=0.85]{sourcesanspro} % Source Sans Pro
    \usepackage{newpxmath} % Compatible math font
}{} % End elegant theme

% Academic theme
\ifdefstring{\fonttheme}{academic}{%
    \usepackage{libertinus} % Libertinus family (serif, sans, and maths)
    \usepackage{libertinust1math} % Matching maths font
}{} % End academic theme

% Scientific theme
\ifdefstring{\fonttheme}{scientific}{%
    \usepackage{stix2} % STIX Two for scientific documents
    \usepackage[scale=0.9]{cabin} % Cabin for sans-serif
    % STIX2 already includes mathematics support
}{} % End scientific theme

% Professional theme
\ifdefstring{\fonttheme}{professional}{%
    \usepackage{newpxtext} % Palatino clone with enhanced features
    \usepackage[scale=0.92]{sourcesanspro} % Source Sans Pro for headings
    \usepackage{newpxmath} % Matching mathematics
}{} % End professional theme

% Contemporary theme
\ifdefstring{\fonttheme}{contemporary}{%
    \usepackage[sfdefault,scaled=.85]{FiraSans} % Modern sans-serif
    \usepackage{newtxsf} % Matching sans-serif maths
}{} % End contemporary theme

% --- FONT SHOWCASE COMMANDS ---
% These commands are for showcasing the font themes in the document.
\newcommand{\showcasemodern}[1]{{\fontfamily{bch}\selectfont #1}}
\newcommand{\showcaseclassic}[1]{{\fontfamily{ppl}\selectfont #1}}
\newcommand{\showcaseelegant}[1]{{\fontfamily{EBGaramond-LF}\selectfont #1}}
\newcommand{\showcaseacademic}[1]{{\fontfamily{LibertinusT1}\selectfont #1}}
\newcommand{\showcasescientific}[1]{{\fontfamily{STIXTwoText-TLF}\selectfont #1}}
\newcommand{\showcaseprofessional}[1]{{\fontfamily{pxr}\selectfont #1}}
\newcommand{\showcasecontemporary}[1]{{\fontfamily{FiraSans-TLF}\selectfont #1}}

% Setting up hyperref for clickable links with subtle styling
\usepackage{hyperref}
\hypersetup{
    colorlinks=true,
    linkcolor=accentcolor, % Use accent colour for links
    urlcolor=accentcolor,
    citecolor=accentcolor,
    pdfborderstyle={/S/U/W 0.5}, % Thinner underline
    % PDF metadata properties
    pdftitle={\notebooktitle},
    pdfauthor={\notebookauthor},
    pdfsubject={\notebooksubtitle},
    pdfcreator={LaTeX with Elegant Notebook Template},
    pdfproducer={LaTeX}
}

% Enhancing lists with custom spacing and symbols
\usepackage{enumitem}
\setlist[itemize]{leftmargin=*, itemsep=0.5ex, topsep=0.5ex, label=\textcolor{accentcolor}{\textbullet}} % Custom bullet
\setlist[enumerate]{leftmargin=*, itemsep=0.5ex, topsep=0.5ex, label=\textcolor{accentcolor}{\arabic*.}}

% Adding line spacing for readability
\usepackage{setspace}
\setstretch{1.15} % Slightly reduced for a denser but still readable look

% Including graphics, table, and caption support
\usepackage{graphicx}
\usepackage{booktabs} % For professional tables
\usepackage{caption}
\captionsetup{font={small,sf}, labelfont={bf,color=accentcolor}, skip=7pt, singlelinecheck=false, format=hang}

% Configuring inline code and environments with tcolorbox
\usepackage[many]{tcolorbox}
\tcbuselibrary{skins,breakable,hooks} % Load tcolorbox libraries for more features

% Improved inline code style
\newtcbox{\inlinecode}{
    nobeforeafter,
    colback=inlinecodebg,
    colframe=inlinecodefg!50,
    boxrule=0.2pt,
    arc=1.0mm,
    left=2pt, right=2pt, top=1.5pt, bottom=1.5pt,
    boxsep=0.7pt,
    on line,
    fontupper=\small\ttfamily\color{inlinecodefg},
    enlarge left by=0mm,
    enlarge right by=0mm
}

% --- ENHANCED TCOLORBOX STYLES ---
\newtcolorbox{notetbox}[1][Note]{
    enhanced, % Use enhanced features
    colback=maincolor!10,
    colframe=maincolor,
    fonttitle=\sffamily\bfseries,
    coltitle=white, % White text on coloured background
    title=#1,
    left=2mm, right=2mm, top=2mm, bottom=2mm,
    arc=1mm,
    breakable % Allow box to break across pages
}

\newtcolorbox{exampletbox}[1][Example]{
    enhanced,
    colback=accentcolor!10,
    colframe=accentcolor,
    fonttitle=\sffamily\bfseries,
    coltitle=white,
    title=#1,
    left=2mm, right=2mm, top=2mm, bottom=2mm,
    arc=2mm,
    breakable
}

% Adding support for code listings
\usepackage{listings}
\lstset{
    basicstyle=\ttfamily\small\color{codefg},
    backgroundcolor=\color{codebg},
    frame=single, % Complete border
    framerule=0.5pt,
    rulecolor=\color{maincolor!50},
    breaklines=true,
    numbers=left,
    numberstyle=\tiny\color{maincolor!70!black},
    keywordstyle=\color{accentcolor}\bfseries,
    commentstyle=\color{maincolor!60!black}\itshape,
    stringstyle=\color{highlightcolor!90!black}, % Style for strings
    identifierstyle=\color{codefg}, % Style for identifiers
    tabsize=2,
    showspaces=false,
    showstringspaces=false,
    captionpos=b, % Position of the caption (bottom)
    aboveskip=1.5em,
    belowskip=1.5em,
    extendedchars=true,
    xleftmargin=18pt, % Increase left margin to ensure enough space for line numbers
    xrightmargin=8pt, % Increase right margin
    numbersep=8pt, % Increase space between line numbers and code
    framexleftmargin=15pt, % Ensure appropriate left frame margin
    framexrightmargin=5pt, % Right frame margin
    framextopmargin=6pt, % Top internal padding
    framexbottommargin=6pt, % Bottom internal padding
    resetmargins=true, % Reset margins
    literate=% Add specific character renderings if needed
        {á}{{\'a}}1 {é}{{\'e}}1 {í}{{\'i}}1 {ó}{{\'o}}1 {ú}{{\'u}}1
        {Á}{{\'A}}1 {É}{{\'E}}1 {Í}{{\'I}}1 {Ó}{{\'O}}1 {Ú}{{\'U}}1
        {ñ}{{\~n}}1 {Ñ}{{\~N}}1
}

% Adding support for simple diagrams with TikZ
\usepackage{tikz}
\usetikzlibrary{shapes.geometric, arrows.meta, positioning, shadows, patterns} % More TikZ libraries

% --- ENHANCED SECTION STYLING ---
\usepackage{titlesec}

% Chapter styling
\titleformat{\chapter}[display]
    {\sffamily\huge\bfseries\color{maincolor}}
    {\textcolor{accentcolor}{\chaptertitlename\ \thechapter}}{1ex}
    {\vspace{0.5ex}\color{maincolor}}
    [\vspace{0.5ex}{\color{maincolor!40}\titlerule[1.5pt]}]

\titleformat{\section}
    {\sffamily\Large\bfseries\color{maincolor}}
    {\thesection}{0.8em}{}
    [\color{maincolor!30}\titlerule] % Adds a rule below the title
\titleformat{\subsection}
    {\sffamily\large\bfseries\color{accentcolor}} % Use accent colour
    {\thesubsection}{0.8em}{}
\titleformat{\subsubsection}
    {\sffamily\normalsize\bfseries\color{accentcolor!80!black}}
    {\thesubsubsection}{0.8em}{}

\titlespacing*{\chapter}{0pt}{0pt}{3ex plus 1ex}
\titlespacing*{\section}{0pt}{2.5ex plus 1ex minus .2ex}{1.5ex plus .2ex}
\titlespacing*{\subsection}{0pt}{2ex plus .8ex minus .2ex}{1ex plus .2ex}
\titlespacing*{\subsubsection}{0pt}{1.5ex plus .6ex minus .2ex}{0.8ex plus .2ex}

% Setting up bibliography with biblatex
\usepackage[backend=biber, style=numeric, sorting=none]{biblatex}
\addbibresource{references.bib} % Reference file

% Creating a custom header and footer
\usepackage{fancyhdr}
\pagestyle{fancy}
\fancyhf{}
\fancyhead[L]{\sffamily\small\textcolor{maincolor!90!black}{\notebooktitle}}
\fancyhead[R]{\sffamily\small\textcolor{maincolor!90!black}{\nouppercase{\leftmark}}} % Section in header
\fancyfoot[C]{\sffamily\small\textcolor{maincolor!90!black}{\thepage}}
\renewcommand{\headrulewidth}{0.5pt}
\renewcommand{\footrulewidth}{0pt} % Remove footer rule for a cleaner look
\renewcommand{\headrule}{\color{maincolor!70}\hrule} 

% Starting the document
\begin{document}

% Generate the title page
% Elegant Notebook - Title Page
% This file defines the layout of the title page

\begin{titlepage}
    \centering
    \begin{tikzpicture}[remember picture, overlay]
        % Use the theme's title page background colour for consistency
        \fill[titlepagebg] (current page.south west) rectangle (current page.north east);
        
        % Minimalist decorative lines for a cleaner, more modern look
        % Top accent line
        \draw[accentcolor, line width=1.5pt]
            ([xshift=2.5cm, yshift=-2.5cm]current page.north west) -- 
            ([xshift=-2.5cm, yshift=-2.5cm]current page.north east);
        
        % Bottom accent line
        \draw[accentcolor, line width=1.5pt]
            ([xshift=2.5cm, yshift=2.5cm]current page.south west) -- 
            ([xshift=-2.5cm, yshift=2.5cm]current page.south east);
    \end{tikzpicture}

    % Use vfill for balanced vertical distribution of content
    \vfill
    
    % Institution - placed at the top for prominence
    {\sffamily\large\textcolor{maincolor!85!black}{\notebookinstitution}}\\[2.5cm]
    
    % Title - central and commanding
    {\sffamily\fontsize{30}{36}\selectfont\bfseries\color{maincolor}\notebooktitle}\\[0.8cm]
    
    % Subtitle - complements the title
    {\sffamily\fontsize{17}{21}\selectfont\color{accentcolor}\notebooksubtitle}
    
    \vfill
    
    % Author block - cleanly grouped
    \begingroup % Group to keep font changes local
    \sffamily
    {\Large\bfseries\color{maincolor!90!black}\notebookauthor}\par\vspace{0.5cm}
    {\normalsize\href{mailto:\notebookemail}{\textcolor{accentcolor}{\notebookemail}}}\par\vspace{0.2cm}
    {\normalsize\color{maincolor!70!black}\notebookdate}
    \endgroup
    
    \vfill
    
    % Footer - unobtrusive and informative
    {\sffamily\small\color{maincolor!70!black}Created with \LaTeX\ using the Elegant Notebook Template}\par
    \vspace*{1.5cm}

\end{titlepage} 

% Adding a table of contents
\tableofcontents
\newpage

% Chapter 1: Introduction
\chapter{Welcome to Your Elegant Notebook}
Welcome to the Elegant Notebook template! We have designed this template to help you create stunning and professional documents with ease. Whether you are taking notes, writing a research paper, or compiling a report, our goal is to make your work shine.

This template is built on a foundation of consistent styling, thoughtful typography, and a powerful theming system.

\begin{notetbox}[Key Features at a Glance]
Here is a glimpse of what you can do:
\begin{itemize}
    \item Craft a unique look with our one-step \textbf{font and colour theme switcher}.
    \item Organise your thoughts with beautifully designed \textbf{custom boxes} for notes and examples.
    \item Showcase your work with elegant support for \textbf{mathematics}, \textbf{code listings}, and \textbf{bibliographies}.
    \item Focus on your content, knowing the design is polished and professional.
\end{itemize}
\end{notetbox}

Let's get started and bring your ideas to life!

% Chapter 2: Getting Started
\chapter{Getting Started in Three Simple Steps}
Getting started is straightforward. Just follow these three steps:

\begin{enumerate}
    \item \textbf{Personalise Your Document}: Open \inlinecode{config/metadata.tex} and add your title, author, and other details.
    \item \textbf{Choose Your Style}: In this file (\inlinecode{elegant-notebook.tex}), find the \inlinecode{\\usetheme} command. Pick your favourite font and colour theme to match your taste. See Chapter \ref{chap:theming} for a full list of themes!
    \item \textbf{Compile Your Notebook}: Run one of the provided scripts to compile your document.
\end{enumerate}

\section{Compiling Your Document}
\label{sec:compilation}
For a quick preview without bibliography processing, use:
\begin{lstlisting}[language=bash]
./simple-compile.sh
\end{lstlisting}

For a full compilation with bibliography support (this requires the \inlinecode{biber} engine), run:
\begin{lstlisting}[language=bash]
./compile.sh
\end{lstlisting}

And that's it! You are ready to start writing.

% Chapter 3: Showcase of Core Features
\chapter{Showcase of Core Features}
\label{chap:features}
This template includes a rich set of features to make your content look its best. Let's explore some of them.

\section{Effortless Typography and Text}
The template provides clean, readable typography right out of the box. You can easily format your text:

\begin{itemize}
    \item Normal text is optimised for readability.
    \item \textbf{Bold text} for giving emphasis.
    \item \textit{Italic text} is perfect for definitions or foreign terms.
    \item \texttt{Monospace text} for technical terms.
    \item \textsf{Sans-serif text} for headings and UI elements.
    \item Of course, you can use \textcolor{accentcolor}{coloured text} and \textcolor{highlightcolor}{highlighted text}, too!
\end{itemize}

Hyperlinks, like this one to the \href{https://www.latex-project.org/}{LaTeX Project}, are styled to fit seamlessly.

\section{Engaging Custom Boxes}
Highlight key information and provide clear examples with our custom box environments.

\subsection{Note Boxes}
Use a `notetbox` to draw attention to important points, warnings, or helpful tips.

\begin{notetbox}
    This is a note box. It is perfect for highlighting information that your reader should not miss.
    
    Note boxes support multiple paragraphs and are styled using your theme's main colour.
\end{notetbox}

\subsection{Example Boxes}
The `exampletbox` is designed to showcase examples, from mathematical proofs to code snippets.

\begin{exampletbox}
    This is an example box, styled with your theme's accent colour.
    
    You can include anything you like, such as a famous equation: $E = mc^2$.
\end{exampletbox}

\section{Beautiful Code Listings}
Whether you are writing a tutorial or documenting your code, our syntax highlighting provides excellent support. You can use \inlinecode{inline code} for small snippets. For larger blocks:

\begin{lstlisting}[language=Python, caption={A simple Python function}, label={lst:factorial}]
def factorial(n):
    """Calculate the factorial of a non-negative integer."""
    if n < 0:
        raise ValueError("Factorial is not defined for negative numbers")
    elif n == 0 or n == 1:
        return 1
    else:
        return n * factorial(n - 1)
        
# Example usage:
result = factorial(5)
print(f"The factorial of 5 is {result}")
\end{lstlisting}

\section{Seamless Mathematics}
The template provides excellent support for mathematical notation, both inline and in display environments.

Inline equations like $\sum_{i=1}^n x_i^2 = x_1^2 + \dots + x_n^2$ fit perfectly within your text. For more complex expressions, use the `align` environment:
\begin{align}
    f(x) &= (x - 1)^2 \\
    f'(x) &= 2(x - 1)
\end{align}

\begin{notetbox}
    Notice how mathematical expressions are rendered clearly in both inline and display forms. The consistent styling ensures your document remains professional and easy to read.
\end{notetbox}

\section{Professional Tables and Figures}
Present your data and diagrams with elegance.

\subsection{Tables}
Tables are styled using the \inlinecode{booktabs} package for a clean, professional look without vertical lines.

\begin{table}[h!]
    \centering
    \caption{Sample Experimental Results}
    \label{tab:results}
    \begin{tabular}{@{}lrr@{}}
        \toprule
        Metric    & Value     & Uncertainty \\
        \midrule
        Mean      & $0.12$    & $\pm 0.03$  \\
        Std. Dev. & $1.05$    & $\pm 0.02$  \\
        Skewness  & $0.05$    & $\pm 0.01$  \\
        \bottomrule
    \end{tabular}
\end{table}

\subsection{Figures and Diagrams}
The template includes full support for TikZ, allowing you to create stunning, customised diagrams.

\begin{figure}[h!]
    \centering
    \begin{tikzpicture}[node distance=2.5cm, auto, >=Latex]
        % Define styles
        \tikzstyle{block} = [rectangle, draw, fill=maincolor!20,
                               text width=7em, text centered, rounded corners, minimum height=3em,
                               drop shadow={opacity=0.7, fill=black!50}]
        \tikzstyle{line} = [draw, -{Stealth[length=2mm, width=1.5mm]}, thick, color=accentcolor]

        % Place nodes
        \node [block] (init) {Initialise};
        \node [block, right=of init] (process) {Process Data};
        \node [block, right=of process] (output) {Output Results};

        % Draw edges
        \path [line] (init) -- (process);
        \path [line] (process) -- (output);
        \path [line, bend left=45] (output) edge node[above, midway, font=\tiny] {Feedback Loop} (init);
    \end{tikzpicture}
    \caption{A styled TikZ diagram showing a simple workflow with a feedback loop.}
    \label{fig:tikzflow}
\end{figure}

\section{Citations and Bibliography}
The template uses BibLaTeX for easy and powerful reference management. You can cite sources from your bibliography, such as \cite{Smith2020} or \cite{Lamport1994}.

\begin{notetbox}
    To use citations, ensure you have the \inlinecode{biber} bibliography processor installed and run the full compilation with \inlinecode{./compile.sh}.
\end{notetbox}

Simply add your references to the \inlinecode{references.bib} file. Here is an example entry:
\begin{lstlisting}[language=TeX, caption={Example reference entry in references.bib}, label={lst:reference}]
@article{Smith2020,
  author  = {Smith, John and Johnson, Sarah},
  title   = {Recent Advances in Document Preparation Systems},
  journal = {Journal of Documentation},
  year    = {2020},
  volume  = {76},
  number  = {3},
  pages   = {710--725},
  doi     = {10.1000/example.doi}
}
\end{lstlisting}

% Chapter 4: Theming Deep Dive
\chapter{Theming: Your Document, Your Style}
\label{chap:theming}
The powerful theming system is one of the standout features of this template. You can effortlessly change the entire look and feel of your document with a single command.

\section{One-Step Theme Switching}
To change your theme, simply edit the \inlinecode{\\usetheme} command at the top of this file. It takes two arguments: the font theme and the colour theme.
\begin{lstlisting}[language=TeX, caption={Using the theme switcher command}, label={lst:themeswitcher}]
% The first parameter is the font theme, the second is the colour theme.
\usetheme{font-theme}{colour-theme}

% Example: Modern fonts with the Oxford Blue theme
\usetheme{modern}{oxford-blue}

% Example: Elegant fonts with the Vintage Rose theme
\usetheme{elegant}{vintage-rose}
\end{lstlisting}

\section{Available Font Themes}
We have curated seven professional font pairings to suit any document. Your current theme is \textbf{\textcolor{accentcolor}{\fonttheme}}.
\begin{itemize}
    \item \textbf{\showcasemodern{modern}}: A contemporary look with Charter and Bera Sans (the default).
    \item \textbf{\showcaseclassic{classic}}: A traditional academic style with Palatino and TeX Gyre Heros.
    \item \textbf{\showcaseelegant{elegant}}: A refined style with EB Garamond and Source Sans Pro.
    \item \textbf{\showcaseacademic{academic}}: The cohesive Libertinus family, perfect for scholarly papers.
    \item \textbf{\showcasescientific{scientific}}: STIX Two and Cabin, optimised for scientific and mathematical content.
    \item \textbf{\showcaseprofessional{professional}}: An enhanced Palatino clone with Source Sans Pro for a smart, business look.
    \item \textbf{\showcasecontemporary{contemporary}}: Modern Fira Sans with matching mathematics support for a clean, unified feel.
\end{itemize}

\section{Available Colour Themes}
Choose from over 17 carefully designed colour themes.
\begin{itemize}
    \item \textbf{Cambridge Family}: \textbf{\textcolor{showcasecambridgeblue}{cambridge-blue}} (default), \textbf{\textcolor{showcasecambridgeclassic}{cambridge-classic}}, \textbf{\textcolor{showcasecambridgeforest}{cambridge-forest}}, \textbf{\textcolor{showcasecambridgesunset}{cambridge-sunset}}, \textbf{\textcolor{showcasecambridgeriver}{cambridge-river}}
    \item \textbf{Oxford Family}: \textbf{\textcolor{showcaseoxfordblue}{oxford-blue}}, \textbf{\textcolor{showcaseoxfordmist}{oxford-mist}}, \textbf{\textcolor{showcaseoxfordmonochrome}{oxford-monochrome}}
    \item \textbf{Sky Family}: \textbf{\textcolor{showcasesummerskies}{summer-skies}}, \textbf{\textcolor{showcaseauroraskies}{aurora-skies}}, \textbf{\textcolor{showcasetwilightskies}{twilight-skies}}, \textbf{\textcolor{showcasemidnightskies}{midnight-skies}}
    \item \textbf{Other Themes}: \textbf{\textcolor{showcasevintagerose}{vintage-rose}}, \textbf{\textcolor{showcaseemerald}{emerald}}, \textbf{\textcolor{showcaseroyalpurple}{royal-purple}}, \textbf{\textcolor{showcasearctic}{arctic}}, \textbf{\textcolor{showcaseautumn}{autumn}}
\end{itemize}

\section{Creating Your Own Colour Theme}
Feeling creative? You can easily define your own colour theme in \inlinecode{config/colours.tex}.

\begin{notetbox}
    To add your own theme, find the "CUSTOM THEME" section in \inlinecode{config/colours.tex} and follow the template pattern provided.
    \begin{itemize}
        \item Choose a unique theme name (e.g., \inlinecode{my-awesome-theme}).
        \item Define all the required colours.
        \item Activate it with \inlinecode{\\usetheme\{font-theme\}\{my-awesome-theme\}}.
    \end{itemize}
\end{notetbox}

Here is an example of what a custom theme definition looks like:
\begin{lstlisting}[language=TeX, caption={Custom theme definition in config/colours.tex}, label={lst:customtheme}]
% Add this block to config/colours.tex
\ifdefstring{\activetheme}{my-custom-theme}{%
    \definecolor{maincolor}{RGB}{76,40,130}       % Royal Purple
    \definecolor{accentcolor}{RGB}{212,175,55}    % Gold
    \definecolor{highlightcolor}{RGB}{220,20,60}  % Crimson
    \definecolor{codebg}{RGB}{245,240,255}        % Light purple background
    \definecolor{codefg}{RGB}{50,30,80}           % Dark purple foreground
    \definecolor{softgray}{RGB}{200,190,210}      % Purple-tinted grey
    \definecolor{titlepagebg}{RGB}{240,235,250}   % Soft purple background
    \definecolor{inlinecodebg}{RGB}{235,230,250}  % Richer purple for inline code
    \definecolor{inlinecodefg}{RGB}{90,50,150}    % Deeper royal purple for inline code
}{}

% Then, in elegant-notebook.tex:
\usetheme{modern}{my-custom-theme}
\end{lstlisting}

% Chapter 5: Advanced Customisation
\chapter{Advanced Customisation}
For those who want to dig deeper, the template offers even more ways to tailor your document.

\section{Creating Custom Box Types}
You can design your own box styles in \inlinecode{config/settings.tex}. For example, you could create a `warningbox` for critical alerts.
\begin{lstlisting}[language=TeX, caption={Creating a custom box type in config/settings.tex}, label={lst:custombox}]
\newtcolorbox{warningbox}{
    enhanced,
    colback=red!10,
    colframe=red!80!black,
    fonttitle=\sffamily\bfseries,
    coltitle=white,
    title=Warning,
    left=2mm, right=2mm, top=2mm, bottom=2mm,
    arc=2mm,
    drop shadow,
    breakable
}
\end{lstlisting}

\section{Understanding the Project Structure}
The template is organised into a clear and logical file structure, making it easy to find what you need.
\begin{lstlisting}[language=bash, caption={Project directory structure}, label={lst:structure}]
elegant-notebook.tex    # Main document (this file)
references.bib          # Bibliography file
README.md               # Project documentation
compile.sh              # Full compilation script
simple-compile.sh       # Basic compilation script
test-themes.tex         # Theme preview tool
config/                 # Configuration files
    colours.tex         # Colour theme definitions
    metadata.tex        # Document metadata (titles, author, etc.)
    settings.tex        # Core package imports and style settings
    theme.tex           # The theme switching system
    title-page.tex      # The title page layout
\end{lstlisting}

% Chapter 6: Conclusion and Best Practices
\chapter{Conclusion and Best Practices}
We hope you enjoy using the Elegant Notebook template. It is designed to be a powerful yet flexible tool that gets out of your way, so you can focus on what matters most: your content.

\begin{notetbox}[Best Practices for Best Results]
Here are a few final tips for getting the best results:
\begin{itemize}
    \item Keep your document organised with a clear hierarchy of chapters and sections.
    \item Use the provided environments (`notetbox', `exampletbox', etc.) consistently for visual coherence.
    \item Choose a font and colour theme that matches the tone and subject matter of your work.
    \item Always use the \inlinecode{./compile.sh} script for the final version to ensure all references and links are correct.
    \item For collaborative projects, we highly recommend using a version control system like Git.
\end{itemize}
\end{notetbox}

Happy writing!

% Section: References
\chapter*{References}
\printbibliography[heading=none]

\end{document} 