% Elegant Notebook Template
% A professional LaTeX template with customisable settings for academic and research work

\documentclass[a4paper,11pt]{article}

% Load theme switching functionality
% Elegant Notebook - Theme Switcher
% This file provides a one-step solution for changing both color and font themes

% Define the theme switcher command
\newcommand{\usetheme}[2]{%
    % Set font theme
    \def\fonttheme{#1}%
    
    % Set color theme
    \def\activetheme{#2}%
}

% Default theme if not specified elsewhere
\providecommand{\fonttheme}{modern}
\providecommand{\activetheme}{cambridge-blue} 

% Select your theme - first parameter is font theme, second is color theme
% Font themes: modern, classic, elegant, academic, scientific, professional, contemporary
% Color themes: cambridge-blue, cambridge-classic, cambridge-forest, cambridge-sunset, cambridge-river,
%              oxford-blue, oxford-mist, oxford-monochrome, summer-skies, aurora-skies, 
%              twilight-skies, midnight-skies, vintage-rose, emerald, royal-purple, arctic, autumn
\usetheme{modern}{cambridge-blue}

% Load the Elegant Notebook settings
% To change font theme, edit the \fonttheme value in config/settings.tex
% Elegant Notebook - Settings File
% This file contains all the package imports and configuration settings

% Load custom metadata
% Elegant Notebook - Metadata
% Customise these values to personalise your notebook

% --- DOCUMENT METADATA ---
% These values will appear on the title page and in headers
\newcommand{\notebooktitle}{Elegant Research Notebook}
\newcommand{\notebooksubtitle}{A Professional \LaTeX{} Template for Academic Writing}
\newcommand{\notebookauthor}{Your Name}
\newcommand{\notebookdate}{\today}
\newcommand{\notebookemail}{your.email@example.com}
\newcommand{\notebookinstitution}{Your Institution}

% Note: PDF metadata properties are automatically set in settings.tex
% after the hyperref package is loaded 

% Load custom colour scheme
% Elegant Notebook - Colour Scheme
% This file defines the colour scheme for the notebook
% Modify these colours to customise the appearance

\usepackage{xcolor}
\usepackage{comment}
\usepackage{etoolbox}

% -----------------------------------------------------
% THEME SELECTION
% -----------------------------------------------------
% The theme can be selected using \usetheme{font-theme}{color-theme}
% Set a default if not already defined
\providecommand{\activetheme}{cambridge-blue}

% -----------------------------------------------------
% THEME DEFINITIONS
% -----------------------------------------------------

% --- COLOUR THEME: CAMBRIDGE BLUE (DEFAULT) ---
\ifx\activetheme\undefined\def\activetheme{cambridge-blue}\fi
\ifdefstring{\activetheme}{cambridge-blue}{%
    % Main theme colours - Official Cambridge Blue (Pantone 557 C)
    \definecolor{maincolor}{RGB}{133,176,154}     % Primary colour (Official Cambridge Blue)
    \definecolor{accentcolor}{RGB}{90,140,115}    % Secondary colour (Darker variant)
    \definecolor{highlightcolor}{RGB}{230,126,34} % Accent colour (Orange) for emphasis

    % Auxiliary colours
    \definecolor{codebg}{RGB}{245,250,247}        % Light background for code
    \definecolor{codefg}{RGB}{45,75,60}           % Darker foreground for code
    \definecolor{softgray}{RGB}{190,210,200}      % Utility grey colour
    \definecolor{titlepagebg}{RGB}{235,245,240}   % Soft background for title page
    \definecolor{inlinecodebg}{RGB}{235,245,240}  % Lighter green-blue background for inline code
    \definecolor{inlinecodefg}{RGB}{0,105,85}     % Slightly darker teal for inline code
}{}

% --- THEME: CAMBRIDGE CLASSIC ---
\ifdefstring{\activetheme}{cambridge-classic}{%
    \definecolor{maincolor}{RGB}{133,176,154}     % Official Cambridge Blue
    \definecolor{accentcolor}{RGB}{64,98,125}     % Cambridge Dark Blue
    \definecolor{highlightcolor}{RGB}{210,180,100}% Cambridge Gold
    \definecolor{codebg}{RGB}{245,248,245}        % Light background for code
    \definecolor{codefg}{RGB}{40,65,80}           % Dark blue-tinted code text
    \definecolor{softgray}{RGB}{180,195,190}      % Blue-green tinted grey
    \definecolor{titlepagebg}{RGB}{235,245,240}   % Soft blue-green background
    \definecolor{inlinecodebg}{RGB}{235,245,242}  % Light blue-green background for inline code
    \definecolor{inlinecodefg}{RGB}{50,90,115}    % Cambridge dark blue for inline code
}{}

% --- THEME: CAMBRIDGE FOREST ---
\ifdefstring{\activetheme}{cambridge-forest}{%
    \definecolor{maincolor}{RGB}{133,176,154}     % Official Cambridge Blue
    \definecolor{accentcolor}{RGB}{65,105,75}     % Deep Forest Green
    \definecolor{highlightcolor}{RGB}{205,170,80} % Autumn Gold
    \definecolor{codebg}{RGB}{245,250,245}        % Light forest-tinted background
    \definecolor{codefg}{RGB}{35,70,45}           % Dark forest green text
    \definecolor{softgray}{RGB}{185,200,185}      % Green-tinted grey
    \definecolor{titlepagebg}{RGB}{240,248,242}   % Soft forest-tinted background
    \definecolor{inlinecodebg}{RGB}{235,245,235}  % Light forest green background for inline code
    \definecolor{inlinecodefg}{RGB}{45,90,55}     % Forest green for inline code
}{}

% --- THEME: CAMBRIDGE SUNSET ---
\ifdefstring{\activetheme}{cambridge-sunset}{%
    \definecolor{maincolor}{RGB}{133,176,154}     % Official Cambridge Blue
    \definecolor{accentcolor}{RGB}{175,95,65}     % Sunset Orange
    \definecolor{highlightcolor}{RGB}{230,180,60} % Golden Sunset
    \definecolor{codebg}{RGB}{250,248,245}        % Warm light background
    \definecolor{codefg}{RGB}{60,60,50}           % Warm dark text
    \definecolor{softgray}{RGB}{200,195,185}      % Warm-tinted grey
    \definecolor{titlepagebg}{RGB}{245,245,240}   % Soft warm background
    \definecolor{inlinecodebg}{RGB}{248,245,240}  % Warm cream background for inline code
    \definecolor{inlinecodefg}{RGB}{160,80,50}    % Sunset orange for inline code
}{}

% --- THEME: CAMBRIDGE RIVER ---
\ifdefstring{\activetheme}{cambridge-river}{%
    \definecolor{maincolor}{RGB}{133,176,154}     % Official Cambridge Blue
    \definecolor{accentcolor}{RGB}{80,130,170}    % River Blue
    \definecolor{highlightcolor}{RGB}{180,210,200}% River Mist
    \definecolor{codebg}{RGB}{240,245,250}        % Cool blue-tinted background
    \definecolor{codefg}{RGB}{40,70,105}          % Deep blue text
    \definecolor{softgray}{RGB}{185,195,205}      % Blue-tinted grey
    \definecolor{titlepagebg}{RGB}{235,242,248}   % Cool light blue background
    \definecolor{inlinecodebg}{RGB}{235,240,248}  % Light blue background for inline code
    \definecolor{inlinecodefg}{RGB}{60,110,150}   % River blue for inline code
}{}

% --- THEME: OXFORD BLUE ---
\ifdefstring{\activetheme}{oxford-blue}{%
    \definecolor{maincolor}{RGB}{0,33,71}        % Oxford Blue
    \definecolor{accentcolor}{RGB}{140,45,25}    % Oxford Red
    \definecolor{highlightcolor}{RGB}{255,205,0} % Gold accent
    \definecolor{codebg}{RGB}{240,240,245}       % Light blue-tinted code background
    \definecolor{codefg}{RGB}{30,30,60}          % Dark blue code foreground
    \definecolor{softgray}{RGB}{190,190,200}     % Blue-tinted grey
    \definecolor{titlepagebg}{RGB}{230,235,245}  % Soft blue background
    \definecolor{inlinecodebg}{RGB}{230,235,245} % Slightly richer blue background for inline code
    \definecolor{inlinecodefg}{RGB}{110,30,20}   % Oxford red for inline code text (using accent colour)
}{}

% --- THEME: OXFORD MIST ---
\ifdefstring{\activetheme}{oxford-mist}{%
    \definecolor{maincolor}{RGB}{0,33,71}         % Oxford Blue
    \definecolor{accentcolor}{RGB}{100,100,130}   % Dusty Blue
    \definecolor{highlightcolor}{RGB}{150,150,170}% Mist Lavender
    \definecolor{codebg}{RGB}{245,245,250}        % Cloudy background
    \definecolor{codefg}{RGB}{40,40,60}           % Slate navy
    \definecolor{softgray}{RGB}{200,200,210}      % Misty grey
    \definecolor{titlepagebg}{RGB}{235,240,245}   % Pale bluish white
    \definecolor{inlinecodebg}{RGB}{235,235,245}  % Richer misty blue for inline code
    \definecolor{inlinecodefg}{RGB}{80,80,120}    % Dusty blue for inline code (from accent)
}{}

% --- THEME: OXFORD MONOCHROME ---
\ifdefstring{\activetheme}{oxford-monochrome}{%
    \definecolor{maincolor}{RGB}{0,33,71}         % Oxford Blue
    \definecolor{accentcolor}{RGB}{50,50,70}      % Charcoal Blue
    \definecolor{highlightcolor}{RGB}{90,90,120}  % Subtle Violet
    \definecolor{codebg}{RGB}{235,235,240}        % Light grey-blue
    \definecolor{codefg}{RGB}{20,30,50}           % Deep navy
    \definecolor{softgray}{RGB}{180,180,190}      % Cool grey
    \definecolor{titlepagebg}{RGB}{225,230,240}   % Very light blue-grey
    \definecolor{inlinecodebg}{RGB}{220,220,235}  % More saturated blue-grey for inline code
    \definecolor{inlinecodefg}{RGB}{40,40,65}     % Deeper charcoal for inline code
}{}

% --- THEME: SUMMER SKIES ---
\ifdefstring{\activetheme}{summer-skies}{%
    \definecolor{maincolor}{RGB}{85,150,205}      % Sky Blue
    \definecolor{accentcolor}{RGB}{55,120,175}    % Deeper Sky Blue
    \definecolor{highlightcolor}{RGB}{240,210,80} % Sunshine Yellow
    \definecolor{codebg}{RGB}{240,248,255}        % Pale sky background
    \definecolor{codefg}{RGB}{30,75,120}          % Deep sky blue text
    \definecolor{softgray}{RGB}{195,210,225}      % Sky-tinted grey
    \definecolor{titlepagebg}{RGB}{235,245,255}   % Soft sky background
    \definecolor{inlinecodebg}{RGB}{230,240,255}  % Light sky blue background for inline code
    \definecolor{inlinecodefg}{RGB}{40,105,160}   % Sky blue for inline code
}{}

% --- THEME: AURORA SKIES ---
\ifdefstring{\activetheme}{aurora-skies}{%
    \definecolor{maincolor}{RGB}{60,125,150}      % Northern Blue
    \definecolor{accentcolor}{RGB}{100,170,120}   % Aurora Green
    \definecolor{highlightcolor}{RGB}{180,140,200}% Aurora Purple
    \definecolor{codebg}{RGB}{240,248,248}        % Cold sky background
    \definecolor{codefg}{RGB}{40,85,100}          % Deep northern blue text
    \definecolor{softgray}{RGB}{185,200,205}      % Northern-tinted grey
    \definecolor{titlepagebg}{RGB}{235,245,245}   % Soft northern sky background
    \definecolor{inlinecodebg}{RGB}{235,245,240}  % Light aurora background for inline code
    \definecolor{inlinecodefg}{RGB}{80,140,100}   % Aurora green for inline code
}{}

% --- THEME: TWILIGHT SKIES ---
\ifdefstring{\activetheme}{twilight-skies}{%
    \definecolor{maincolor}{RGB}{90,100,160}      % Twilight Blue
    \definecolor{accentcolor}{RGB}{150,100,150}   % Twilight Purple
    \definecolor{highlightcolor}{RGB}{230,180,140}% Sunset Gold
    \definecolor{codebg}{RGB}{242,242,250}        % Cool evening background
    \definecolor{codefg}{RGB}{60,60,100}          % Deep twilight text
    \definecolor{softgray}{RGB}{190,190,210}      % Twilight-tinted grey
    \definecolor{titlepagebg}{RGB}{235,235,245}   % Soft twilight background
    \definecolor{inlinecodebg}{RGB}{235,230,245}  % Light twilight background for inline code
    \definecolor{inlinecodefg}{RGB}{130,85,135}   % Twilight purple for inline code
}{}

% --- THEME: MIDNIGHT SKIES ---
\ifdefstring{\activetheme}{midnight-skies}{%
    \definecolor{maincolor}{RGB}{45,60,110}      % Midnight Blue
    \definecolor{accentcolor}{RGB}{90,95,145}    % Softer Midnight Blue
    \definecolor{highlightcolor}{RGB}{180,185,220} % Starlight Blue
    \definecolor{codebg}{RGB}{240,240,248}       % Deep night sky background
    \definecolor{codefg}{RGB}{35,45,85}          % Deep midnight text
    \definecolor{softgray}{RGB}{180,180,200}     % Night-tinted grey
    \definecolor{titlepagebg}{RGB}{232,232,242}  % Soft midnight background
    \definecolor{inlinecodebg}{RGB}{230,230,245} % Light midnight background for inline code
    \definecolor{inlinecodefg}{RGB}{70,80,130}   % Midnight blue for inline code
}{}

% --- THEME: VINTAGE ROSE ---
\ifdefstring{\activetheme}{vintage-rose}{%
    \definecolor{maincolor}{RGB}{196,146,151}     % Dusty Rose (Main colour)
    \definecolor{accentcolor}{RGB}{156,110,115}   % Muted Berry (Secondary colour)
    \definecolor{highlightcolor}{RGB}{242,206,184}% Pale Peach (Highlight)
    \definecolor{codebg}{RGB}{253,248,249}        % Light Pinkish Cream (Code background)
    \definecolor{codefg}{RGB}{105,73,77}          % Brownish Purple (Code foreground)
    \definecolor{softgray}{RGB}{207,192,194}      % Rose-tinted Grey (Utility grey)
    \definecolor{titlepagebg}{RGB}{249,240,241}   % Soft Rose White (Title page background)
    \definecolor{inlinecodebg}{RGB}{248,237,240}  % Deeper rose tint for inline code
    \definecolor{inlinecodefg}{RGB}{140,90,95}    % Richer berry tone for inline code
}{}

% --- THEME: EMERALD ---
\ifdefstring{\activetheme}{emerald}{%
    \definecolor{maincolor}{RGB}{0,112,74}        % Emerald Green
    \definecolor{accentcolor}{RGB}{52,86,125}     % Ocean Blue
    \definecolor{highlightcolor}{RGB}{237,169,33} % Golden Yellow
    \definecolor{codebg}{RGB}{240,248,245}        % Light green-tinted code background
    \definecolor{codefg}{RGB}{20,60,40}           % Dark green code foreground
    \definecolor{softgray}{RGB}{180,200,190}      % Green-tinted grey
    \definecolor{titlepagebg}{RGB}{230,245,240}   % Soft green background
    \definecolor{inlinecodebg}{RGB}{232,245,238}  % Richer emerald tint for inline code
    \definecolor{inlinecodefg}{RGB}{0,90,60}      % Deep emerald for inline code text
}{}

% --- THEME: ROYAL PURPLE ---
\ifdefstring{\activetheme}{royal-purple}{%
    \definecolor{maincolor}{RGB}{76,40,130}       % Royal Purple
    \definecolor{accentcolor}{RGB}{212,175,55}    % Gold
    \definecolor{highlightcolor}{RGB}{220,20,60}  % Crimson
    \definecolor{codebg}{RGB}{245,240,255}        % Light purple background
    \definecolor{codefg}{RGB}{50,30,80}           % Dark purple foreground
    \definecolor{softgray}{RGB}{200,190,210}      % Purple-tinted grey
    \definecolor{titlepagebg}{RGB}{240,235,250}   % Soft purple background
    \definecolor{inlinecodebg}{RGB}{235,230,250}  % Richer purple for inline code
    \definecolor{inlinecodefg}{RGB}{90,50,150}    % Deeper royal purple for inline code
}{}

% --- THEME: ARCTIC ---
\ifdefstring{\activetheme}{arctic}{%
    \definecolor{maincolor}{RGB}{64,103,129}      % Arctic Blue
    \definecolor{accentcolor}{RGB}{25,64,86}      % Deep Arctic
    \definecolor{highlightcolor}{RGB}{180,210,220}% Ice Blue
    \definecolor{codebg}{RGB}{235,245,250}        % Snow White
    \definecolor{codefg}{RGB}{40,60,80}           % Midnight Blue
    \definecolor{softgray}{RGB}{190,205,215}      % Glacier Grey
    \definecolor{titlepagebg}{RGB}{230,240,245}   % Frost White
    \definecolor{inlinecodebg}{RGB}{225,238,248}  % Deeper ice blue for inline code
    \definecolor{inlinecodefg}{RGB}{25,80,105}    % Richer arctic blue for inline code
}{}

% --- THEME: AUTUMN ---
\ifdefstring{\activetheme}{autumn}{%
    \definecolor{maincolor}{RGB}{165,82,42}       % Sienna Brown
    \definecolor{accentcolor}{RGB}{205,133,63}    % Peru
    \definecolor{highlightcolor}{RGB}{220,105,30} % Burnt Orange
    \definecolor{codebg}{RGB}{252,248,240}        % Cream background
    \definecolor{codefg}{RGB}{80,40,20}           % Dark Brown foreground
    \definecolor{softgray}{RGB}{210,200,180}      % Beige Grey
    \definecolor{titlepagebg}{RGB}{250,245,235}   % Pale Cream
    \definecolor{inlinecodebg}{RGB}{248,242,233}  % Richer autumn gold for inline code
    \definecolor{inlinecodefg}{RGB}{170,90,45}    % Deeper autumn tone for inline code
}{}

% -----------------------------------------------------
% CUSTOM THEME 
% -----------------------------------------------------
% Create your own theme by adding a new selection block:
%
% \ifdefstring{\activetheme}{your-theme-name}{%
%     \definecolor{maincolor}{RGB}{0,0,0}           % Primary colour
%     \definecolor{accentcolor}{RGB}{0,0,0}         % Secondary colour
%     \definecolor{highlightcolor}{RGB}{0,0,0}      % Accent colour for emphasis
%     \definecolor{codebg}{RGB}{0,0,0}              % Background for code blocks
%     \definecolor{codefg}{RGB}{0,0,0}              % Text for code blocks
%     \definecolor{softgray}{RGB}{0,0,0}            % Utility grey colour
%     \definecolor{titlepagebg}{RGB}{0,0,0}         % Background for title page
%     \definecolor{inlinecodebg}{RGB}{0,0,0}        % Background for inline code
%     \definecolor{inlinecodefg}{RGB}{0,0,0}        % Text colour for inline code
% }{}

% Setting up geometry for balanced margins
\usepackage[margin=0.8in, headheight=14pt]{geometry} % Increased headheight to prevent fancyhdr warnings

% Including essential packages for language, encoding, and functionality
\usepackage[utf8]{inputenc}
\usepackage[T1]{fontenc}
\usepackage[british]{babel} % British English spelling and conventions
\usepackage{csquotes} % Recommended when using babel with biblatex

% Adding mathematical packages first to avoid conflicts
\usepackage{amsmath}
\usepackage{amssymb}

% --- FONT SELECTION ---
% The font theme is now set using the \usetheme command in elegant-notebook.tex
% \fonttheme is defined by the theme system, so we just need to use it
\usepackage{microtype} % Improved typography for all fonts
\usepackage[varqu,varl]{inconsolata} % Better monospace base font
\usepackage{etoolbox} % Required for string comparison

% Save definition of \Bbbk from amssymb to avoid redefinition errors
\let\amssymbBbbk\Bbbk

% Modern theme (default)
\ifx\fonttheme\@empty\def\fonttheme{modern}\fi
\ifdefstring{\fonttheme}{modern}{%
    \usepackage{charter} % Serif
    \usepackage[scaled=0.85]{berasans} % Sans-serif
    \usepackage{newtxmath} % Compatible mathematics font
}{} % End modern theme

% Restore original \Bbbk definition
\let\Bbbk\amssymbBbbk

% Classic theme
\ifdefstring{\fonttheme}{classic}{%
    \usepackage{palatino} % Palatino for serif
    \usepackage[scale=0.9]{tgheros} % TeX Gyre Heros for sans-serif
    \usepackage{newpxmath} % Palatino-compatible mathematics
}{} % End classic theme

% Elegant theme
\ifdefstring{\fonttheme}{elegant}{%
    \usepackage{ebgaramond} % Garamond for elegant serif
    \usepackage[scale=0.85]{sourcesanspro} % Source Sans Pro
    \usepackage{newpxmath} % Compatible math font
}{} % End elegant theme

% Academic theme
\ifdefstring{\fonttheme}{academic}{%
    \usepackage{libertinus} % Libertinus family (serif, sans, and maths)
    \usepackage{libertinust1math} % Matching maths font
}{} % End academic theme

% Scientific theme
\ifdefstring{\fonttheme}{scientific}{%
    \usepackage{stix2} % STIX Two for scientific documents
    \usepackage[scale=0.9]{cabin} % Cabin for sans-serif
    % STIX2 already includes mathematics support
}{} % End scientific theme

% Professional theme
\ifdefstring{\fonttheme}{professional}{%
    \usepackage{newpxtext} % Palatino clone with enhanced features
    \usepackage[scale=0.92]{sourcesanspro} % Source Sans Pro for headings
    \usepackage{newpxmath} % Matching mathematics
}{} % End professional theme

% Contemporary theme
\ifdefstring{\fonttheme}{contemporary}{%
    \usepackage[sfdefault,scaled=.85]{FiraSans} % Modern sans-serif
    \usepackage{newtxsf} % Matching sans-serif maths
}{} % End contemporary theme

% --- FONT SHOWCASE COMMANDS ---
% These commands are for showcasing the font themes in the document.
\newcommand{\showcasemodern}[1]{{\fontfamily{bch}\selectfont #1}}
\newcommand{\showcaseclassic}[1]{{\fontfamily{ppl}\selectfont #1}}
\newcommand{\showcaseelegant}[1]{{\fontfamily{EBGaramond-LF}\selectfont #1}}
\newcommand{\showcaseacademic}[1]{{\fontfamily{LibertinusT1}\selectfont #1}}
\newcommand{\showcasescientific}[1]{{\fontfamily{STIXTwoText-TLF}\selectfont #1}}
\newcommand{\showcaseprofessional}[1]{{\fontfamily{pxr}\selectfont #1}}
\newcommand{\showcasecontemporary}[1]{{\fontfamily{FiraSans-TLF}\selectfont #1}}

% Setting up hyperref for clickable links with subtle styling
\usepackage{hyperref}
\hypersetup{
    colorlinks=true,
    linkcolor=accentcolor, % Use accent colour for links
    urlcolor=accentcolor,
    citecolor=accentcolor,
    pdfborderstyle={/S/U/W 0.5}, % Thinner underline
    % PDF metadata properties
    pdftitle={\notebooktitle},
    pdfauthor={\notebookauthor},
    pdfsubject={\notebooksubtitle},
    pdfcreator={LaTeX with Elegant Notebook Template},
    pdfproducer={LaTeX}
}

% Enhancing lists with custom spacing and symbols
\usepackage{enumitem}
\setlist[itemize]{leftmargin=*, itemsep=0.5ex, topsep=0.5ex, label=\textcolor{accentcolor}{\textbullet}} % Custom bullet
\setlist[enumerate]{leftmargin=*, itemsep=0.5ex, topsep=0.5ex, label=\textcolor{accentcolor}{\arabic*.}}

% Adding line spacing for readability
\usepackage{setspace}
\setstretch{1.15} % Slightly reduced for a denser but still readable look

% Including graphics, table, and caption support
\usepackage{graphicx}
\usepackage{booktabs} % For professional tables
\usepackage{caption}
\captionsetup{font={small,sf}, labelfont={bf,color=accentcolor}, skip=7pt, singlelinecheck=false, format=hang}

% Configuring inline code and environments with tcolorbox
\usepackage[many]{tcolorbox}
\tcbuselibrary{skins,breakable,hooks} % Load tcolorbox libraries for more features

% Improved inline code style
\newtcbox{\inlinecode}{
    nobeforeafter,
    colback=inlinecodebg,
    colframe=inlinecodefg!50,
    boxrule=0.2pt,
    arc=1.0mm,
    left=2pt, right=2pt, top=1.5pt, bottom=1.5pt,
    boxsep=0.7pt,
    on line,
    fontupper=\small\ttfamily\color{inlinecodefg},
    enlarge left by=0mm,
    enlarge right by=0mm
}

% --- ENHANCED TCOLORBOX STYLES ---
\newtcolorbox{notetbox}[1][Note]{
    enhanced, % Use enhanced features
    colback=maincolor!10,
    colframe=maincolor,
    fonttitle=\sffamily\bfseries,
    coltitle=white, % White text on coloured background
    title=#1,
    left=2mm, right=2mm, top=2mm, bottom=2mm,
    arc=1mm,
    breakable % Allow box to break across pages
}

\newtcolorbox{exampletbox}[1][Example]{
    enhanced,
    colback=accentcolor!10,
    colframe=accentcolor,
    fonttitle=\sffamily\bfseries,
    coltitle=white,
    title=#1,
    left=2mm, right=2mm, top=2mm, bottom=2mm,
    arc=2mm,
    breakable
}

% Adding support for code listings
\usepackage{listings}
\lstset{
    basicstyle=\ttfamily\small\color{codefg},
    backgroundcolor=\color{codebg},
    frame=single, % Complete border
    framerule=0.5pt,
    rulecolor=\color{maincolor!50},
    breaklines=true,
    numbers=left,
    numberstyle=\tiny\color{maincolor!70!black},
    keywordstyle=\color{accentcolor}\bfseries,
    commentstyle=\color{maincolor!60!black}\itshape,
    stringstyle=\color{highlightcolor!90!black}, % Style for strings
    identifierstyle=\color{codefg}, % Style for identifiers
    tabsize=2,
    showspaces=false,
    showstringspaces=false,
    captionpos=b, % Position of the caption (bottom)
    aboveskip=1.5em,
    belowskip=1.5em,
    extendedchars=true,
    xleftmargin=18pt, % Increase left margin to ensure enough space for line numbers
    xrightmargin=8pt, % Increase right margin
    numbersep=8pt, % Increase space between line numbers and code
    framexleftmargin=15pt, % Ensure appropriate left frame margin
    framexrightmargin=5pt, % Right frame margin
    framextopmargin=6pt, % Top internal padding
    framexbottommargin=6pt, % Bottom internal padding
    resetmargins=true, % Reset margins
    literate=% Add specific character renderings if needed
        {á}{{\'a}}1 {é}{{\'e}}1 {í}{{\'i}}1 {ó}{{\'o}}1 {ú}{{\'u}}1
        {Á}{{\'A}}1 {É}{{\'E}}1 {Í}{{\'I}}1 {Ó}{{\'O}}1 {Ú}{{\'U}}1
        {ñ}{{\~n}}1 {Ñ}{{\~N}}1
}

% Adding support for simple diagrams with TikZ
\usepackage{tikz}
\usetikzlibrary{shapes.geometric, arrows.meta, positioning, shadows, patterns} % More TikZ libraries

% --- ENHANCED SECTION STYLING ---
\usepackage{titlesec}

% Chapter styling
\titleformat{\chapter}[display]
    {\sffamily\huge\bfseries\color{maincolor}}
    {\textcolor{accentcolor}{\chaptertitlename\ \thechapter}}{1ex}
    {\vspace{0.5ex}\color{maincolor}}
    [\vspace{0.5ex}{\color{maincolor!40}\titlerule[1.5pt]}]

\titleformat{\section}
    {\sffamily\Large\bfseries\color{maincolor}}
    {\thesection}{0.8em}{}
    [\color{maincolor!30}\titlerule] % Adds a rule below the title
\titleformat{\subsection}
    {\sffamily\large\bfseries\color{accentcolor}} % Use accent colour
    {\thesubsection}{0.8em}{}
\titleformat{\subsubsection}
    {\sffamily\normalsize\bfseries\color{accentcolor!80!black}}
    {\thesubsubsection}{0.8em}{}

\titlespacing*{\chapter}{0pt}{0pt}{3ex plus 1ex}
\titlespacing*{\section}{0pt}{2.5ex plus 1ex minus .2ex}{1.5ex plus .2ex}
\titlespacing*{\subsection}{0pt}{2ex plus .8ex minus .2ex}{1ex plus .2ex}
\titlespacing*{\subsubsection}{0pt}{1.5ex plus .6ex minus .2ex}{0.8ex plus .2ex}

% Setting up bibliography with biblatex
\usepackage[backend=biber, style=numeric, sorting=none]{biblatex}
\addbibresource{references.bib} % Reference file

% Creating a custom header and footer
\usepackage{fancyhdr}
\pagestyle{fancy}
\fancyhf{}
\fancyhead[L]{\sffamily\small\textcolor{maincolor!90!black}{\notebooktitle}}
\fancyhead[R]{\sffamily\small\textcolor{maincolor!90!black}{\nouppercase{\leftmark}}} % Section in header
\fancyfoot[C]{\sffamily\small\textcolor{maincolor!90!black}{\thepage}}
\renewcommand{\headrulewidth}{0.5pt}
\renewcommand{\footrulewidth}{0pt} % Remove footer rule for a cleaner look
\renewcommand{\headrule}{\color{maincolor!70}\hrule} 

% Starting the document
\begin{document}

% Generate the title page
% Elegant Notebook - Title Page
% This file defines the layout of the title page

\begin{titlepage}
    \centering
    \begin{tikzpicture}[remember picture, overlay]
        % Use the theme's title page background colour for consistency
        \fill[titlepagebg] (current page.south west) rectangle (current page.north east);
        
        % Minimalist decorative lines for a cleaner, more modern look
        % Top accent line
        \draw[accentcolor, line width=1.5pt]
            ([xshift=2.5cm, yshift=-2.5cm]current page.north west) -- 
            ([xshift=-2.5cm, yshift=-2.5cm]current page.north east);
        
        % Bottom accent line
        \draw[accentcolor, line width=1.5pt]
            ([xshift=2.5cm, yshift=2.5cm]current page.south west) -- 
            ([xshift=-2.5cm, yshift=2.5cm]current page.south east);
    \end{tikzpicture}

    % Use vfill for balanced vertical distribution of content
    \vfill
    
    % Institution - placed at the top for prominence
    {\sffamily\large\textcolor{maincolor!85!black}{\notebookinstitution}}\\[2.5cm]
    
    % Title - central and commanding
    {\sffamily\fontsize{30}{36}\selectfont\bfseries\color{maincolor}\notebooktitle}\\[0.8cm]
    
    % Subtitle - complements the title
    {\sffamily\fontsize{17}{21}\selectfont\color{accentcolor}\notebooksubtitle}
    
    \vfill
    
    % Author block - cleanly grouped
    \begingroup % Group to keep font changes local
    \sffamily
    {\Large\bfseries\color{maincolor!90!black}\notebookauthor}\par\vspace{0.5cm}
    {\normalsize\href{mailto:\notebookemail}{\textcolor{accentcolor}{\notebookemail}}}\par\vspace{0.2cm}
    {\normalsize\color{maincolor!70!black}\notebookdate}
    \endgroup
    
    \vfill
    
    % Footer - unobtrusive and informative
    {\sffamily\small\color{maincolor!70!black}Created with \LaTeX\ using the Elegant Notebook Template}\par
    \vspace*{1.5cm}

\end{titlepage} 

% Adding a table of contents
\tableofcontents
\newpage

% Section: Introduction
\section{Introduction}
Welcome to the Elegant Notebook LaTeX template. This template is designed to create beautiful, professional-looking documents with minimal effort. It features consistent styling, thoughtful typography, and useful environments for academic and technical writing.

This section demonstrates the basic typography and layout of the template, including:

\begin{itemize}
    \item Regular paragraphs with optimised line spacing and margins
    \item \textbf{Bold text} for emphasis and \textit{italic text} for foreign terms
    \item Hyperlinks like \href{https://www.latex-project.org/}{LaTeX Project}
    \item Mathematical expressions: $E = mc^2$ and $\sum_{i=1}^{n} i = \frac{n(n+1)}{2}$
\end{itemize}

\subsection{Objectives}
\begin{itemize}
    \item Organise notes in a professional and structured format
    \item Include mathematical equations, code, and references seamlessly
    \item Provide reusable environments for notes and examples
\end{itemize}

% Section: Getting Started
\section{Getting Started}
To use this template effectively:

\begin{enumerate}
    \item Edit \inlinecode{config/metadata.tex} to customise your document title, author, institution, etc.
    \item Choose a colour theme in \inlinecode{config/colours.tex} or use the default
    \item Select a font theme by changing the \inlinecode{\\fonttheme} value in \inlinecode{config/settings.tex}
    \item Start writing your content in this file, using the provided environments and styles
    \item Compile using the included scripts (see Section \ref{sec:compilation})
\end{enumerate}

\subsection{Compilation}
\label{sec:compilation}
For basic compilation without bibliography processing:

\begin{lstlisting}[language=bash]
./simple-compile.sh
\end{lstlisting}

For full compilation with bibliography support (requires \inlinecode{biber}):

\begin{lstlisting}[language=bash]
./compile.sh
\end{lstlisting}

% Section: Template Features
\section{Template Features}
The template includes several features to make your document look professional:

\begin{itemize}
    \item \textbf{Consistent styling} for headings, lists, and other elements
    \item \textbf{Custom colour schemes} that you can easily modify
    \item \textbf{Font themes} with professionally curated typography pairings
    \item \textbf{Styled boxes} for notes, examples, and other content
    \item \textbf{Code listings} with syntax highlighting
    \item \textbf{Mathematical support} through AMS-LaTeX
    \item \textbf{Bibliography management} with BibLaTeX
\end{itemize}

\subsection{Custom Box Environments}
The template provides several custom box environments for different purposes:

\subsubsection{Note Boxes}
Note boxes can be used to highlight important information:

\begin{notetbox}
    Important information goes here. 
    
    Multiple paragraphs are supported. Note boxes are perfect for emphasising key points, warnings, or special instructions.
\end{notetbox}

\subsubsection{Example Boxes}
Example boxes are meant for demonstrating concepts:

\begin{exampletbox}
    This is an example box.
    
    It uses the accent colour from your selected theme.
    
    You can include mathematics: $E = mc^2$
\end{exampletbox}

\subsection{Mathematics Support}
The template provides excellent support for mathematical notation:

\begin{align}
    f(x) &= x^2 - 2x + 1 \\
    f'(x) &= 2x - 2
\end{align}

Inline equations like $\sum_{i=1}^n x_i^2 = x_1^2 + x_2^2 + \dots + x_n^2$ are also well-supported.

\begin{notetbox}
    The summation formula above is useful for calculating the sum of squares in statistical analyses.

    Note how mathematical expressions are rendered clearly both inline and in displayed form.
\end{notetbox}

\subsection{Code Listings}
The template provides support for code listings with syntax highlighting:

\begin{lstlisting}[language=Python, caption={Simple Python function}, label={lst:factorial}]
def factorial(n):
    """Calculate the factorial of n."""
    if n == 0 or n == 1:
        return 1
    else:
        return n * factorial(n-1)
        
# Calculate factorial of 5
result = factorial(5)
print(f"Factorial of 5 is {result}")
\end{lstlisting}

You can also use \inlinecode{inline code} for short snippets within text.

For data analysis examples:

\begin{lstlisting}[language=Python, caption={Python script for data generation}, label={lst:data}]
import numpy as np

# Generate random data
def generate_data(size=100):
    data = np.random.randn(size) # Generate random data
    mean_val = np.mean(data)
    print(f"Mean: {mean_val:.2f}") # Print the mean
    return data

my_data = generate_data()
string_example = "This is a string in Python."
\end{lstlisting}

\begin{exampletbox}
    The above code (Listing \ref{lst:data}) demonstrates random data generation using \inlinecode{numpy}.

    Adjust the sample size as needed. The styling of the code box is automatically determined by the active colour theme.
\end{exampletbox}

% Section: Typography Examples
\section{Typography Examples}

This section showcases the typography of the template, demonstrating the different elements and their styling.

\subsection{Available Font Themes}

The template includes seven professionally designed font themes that you can select by changing the \inlinecode{\\fonttheme} value in \inlinecode{config/settings.tex}:

\begin{center}
\large\textbf{Current Font Theme: \textcolor{accentcolor}{\fonttheme}}
\end{center}

\begin{itemize}
    \item \textbf{modern} - Contemporary look with Charter and Bera Sans (default)
    \item \textbf{classic} - Traditional academic style with Palatino and TeX Gyre Heros
    \item \textbf{elegant} - Refined style with EB Garamond and Source Sans Pro
    \item \textbf{academic} - Cohesive Libertinus family for academic papers
    \item \textbf{scientific} - STIX Two and Cabin optimised for scientific content
    \item \textbf{professional} - Enhanced Palatino clone with Source Sans Pro for a business look
    \item \textbf{contemporary} - Modern Fira Sans with matching mathematics support
\end{itemize}

You can preview different themes by changing the \inlinecode{\\fonttheme} value in \inlinecode{config/settings.tex} and recompiling.

\subsection{Text Formatting}

Here are examples of text formatting available in the template:

\begin{itemize}
    \item Normal text in the main body font
    \item \textbf{Bold text} for emphasis
    \item \textit{Italic text} for foreign terms or titles
    \item \texttt{Monospace text} for code or technical terms
    \item \textsf{Sans-serif text} for UI elements or headings
    \item \textcolor{accentcolor}{Coloured text} using the accent colour
    \item \textcolor{highlightcolor}{Highlighted text} using the highlight colour
\end{itemize}

% Section: Custom Themes
\section{Custom Themes}
The template comes with several built-in colour themes, and you can also create your own.

\subsection{Built-in Themes}
The template includes the following themes:
\begin{itemize}
    \item \textbf{Cambridge Blue} (default) - The official University of Cambridge blue (Pantone 557 C)
    \item \textbf{Cambridge} variants:
    \begin{itemize}
        \item \textbf{Cambridge Classic} - Cambridge blue with dark blue and gold accents
        \item \textbf{Cambridge Forest} - Cambridge blue with forest green and autumn gold accents
        \item \textbf{Cambridge Sunset} - Cambridge blue with sunset orange and golden accents
        \item \textbf{Cambridge River} - Cambridge blue with river blue and mist accents
    \end{itemize}
    \item \textbf{Oxford} themes:
    \begin{itemize}
        \item \textbf{Oxford Blue} - Deep Oxford blue with Oxford red and gold accents
        \item \textbf{Oxford Mist} - Oxford blue with dusty blue and mist lavender
        \item \textbf{Oxford Monochrome} - Oxford blue with charcoal and subtle violet tones
    \end{itemize}
    \item \textbf{Sky} themes:
    \begin{itemize}
        \item \textbf{Summer Skies} - Bright sky blue with sunshine yellow accents
        \item \textbf{Twilight Skies} - Twilight blue with purple and sunset gold
        \item \textbf{Midnight Skies} - Deep midnight blue with starlight accents
        \item \textbf{Aurora Skies} - Northern blue with aurora green and purple
    \end{itemize}
    \item \textbf{Other themes}:
    \begin{itemize}
        \item \textbf{Vintage Rose} - Dusty rose with muted berry and peach highlights
        \item \textbf{Emerald} - Rich emerald green with ocean blue and golden accents
        \item \textbf{Royal Purple} - Royal purple with gold and crimson accents
        \item \textbf{Arctic} - Arctic blue with ice blue highlights
        \item \textbf{Autumn} - Warm sienna and burnt orange tones with cream backgrounds
    \end{itemize}
\end{itemize}

\subsection{One-Step Theme Switching}
The template now features a simplified theme switching system. To change both font and colour themes at once, use the \inlinecode{\\usetheme} command near the top of your main document:

\begin{lstlisting}[language=TeX, caption={Using the theme switcher command}, label={lst:themeswitcher}]
% Select your theme - first parameter is font theme, second is color theme
\usetheme{font-theme}{color-theme}

% Example: Modern fonts with Oxford Blue theme
\usetheme{modern}{oxford-blue}

% Example: Elegant fonts with Vintage Rose theme
\usetheme{elegant}{vintage-rose}
\end{lstlisting}

The available font themes are:
\begin{itemize}
    \item \textbf{modern} - Contemporary look with Charter and Bera Sans (default)
    \item \textbf{classic} - Traditional academic style with Palatino and TeX Gyre Heros
    \item \textbf{elegant} - Refined style with EB Garamond and Source Sans Pro
    \item \textbf{academic} - Cohesive Libertinus family for academic papers
    \item \textbf{scientific} - STIX Two and Cabin optimised for scientific content
    \item \textbf{professional} - Enhanced Palatino clone with Source Sans Pro for a business look
    \item \textbf{contemporary} - Modern Fira Sans with matching mathematics support
\end{itemize}

The available colour themes are:
\begin{itemize}
    \item Cambridge family: \textbf{cambridge-blue} (default), \textbf{cambridge-classic}, \textbf{cambridge-forest}, \textbf{cambridge-sunset}, \textbf{cambridge-river}
    \item Oxford family: \textbf{oxford-blue}, \textbf{oxford-mist}, \textbf{oxford-monochrome}
    \item Sky family: \textbf{summer-skies}, \textbf{aurora-skies}, \textbf{twilight-skies}, \textbf{midnight-skies}
    \item Other themes: \textbf{vintage-rose}, \textbf{emerald}, \textbf{royal-purple}, \textbf{arctic}, \textbf{autumn}
\end{itemize}

\subsection{Creating Custom Themes}
You can create your own theme by defining a new theme in \inlinecode{config/colours.tex}:

\begin{notetbox}
    To add a custom colour theme, locate the "CUSTOM THEME" section in \inlinecode{config/colours.tex} and add your theme using the provided template pattern.
    \begin{itemize}
        \item Choose a unique theme name like \inlinecode{my-theme-name}
        \item Define all the required colours
        \item Use it with \inlinecode{\\usetheme\{font-theme\}\{my-theme-name\}}
    \end{itemize}
\end{notetbox}

\begin{lstlisting}[language=TeX, caption={Custom theme definition in config/colours.tex}, label={lst:customtheme}]
% In config/colours.tex:
\ifdefstring{\activetheme}{my-custom-theme}{%
    \definecolor{maincolor}{RGB}{76,40,130}       % Royal Purple
    \definecolor{accentcolor}{RGB}{212,175,55}    % Gold
    \definecolor{highlightcolor}{RGB}{220,20,60}  % Crimson
    \definecolor{codebg}{RGB}{245,240,255}        % Light purple background
    \definecolor{codefg}{RGB}{50,30,80}           % Dark purple foreground
    \definecolor{softgray}{RGB}{200,190,210}      % Purple-tinted grey
    \definecolor{titlepagebg}{RGB}{240,235,250}   % Soft purple background
    \definecolor{inlinecodebg}{RGB}{235,230,250}  % Richer purple for inline code
    \definecolor{inlinecodefg}{RGB}{90,50,150}    % Deeper royal purple for inline code
}{}

% Then in elegant-notebook.tex:
\usetheme{modern}{my-custom-theme}
\end{lstlisting}

\subsection{Colour Selection Principles}
When choosing colours for your academic document, consider these research-supported principles:

\begin{itemize}
    \item \textcolor{blue}{Blue tones} convey professionalism, trustworthiness and clarity—suitable for scientific and technical content
    \item \textcolor{green!70!black}{Green shades} suggest growth, balance and environmental themes—effective for sustainability research
    \item \textcolor{violet}{Purple hues} represent creativity, wisdom and quality—appropriate for arts and humanities
    \item \textcolor{red}{Red accents} can emphasise critical information, but should be used sparingly in academic contexts
\end{itemize}

Aim for a balanced colour palette with sufficient contrast between text and background to maintain readability while reflecting your document's subject matter appropriately.

% Section: Tables and Figures
\section{Tables and Figures}
The template provides elegant styling for tables and figures.

\subsection{Tables}
Tables are styled with the \inlinecode{booktabs} package for professional appearance:

\begin{table}[h!]
    \centering
    \caption{Sample Results with Enhanced Styling}
    \label{tab:results}
    \begin{tabular}{@{}lrr@{}}
        \toprule
        Metric    & Value     & Uncertainty \\
        \midrule
        Mean      & $0.12$    & $\pm 0.03$  \\
        Std. Dev. & $1.05$    & $\pm 0.02$  \\
        Skewness  & $0.05$    & $\pm 0.01$  \\
        \bottomrule
    \end{tabular}
\end{table}

\subsection{Figures and Diagrams}
The template includes support for TikZ diagrams:

\begin{figure}[h!]
    \centering
    \begin{tikzpicture}[node distance=2cm, auto, >=Latex]
        % Define styles
        \tikzstyle{block} = [rectangle, draw, fill=maincolor!20,
                               text width=6em, text centered, rounded corners, minimum height=3em,
                               drop shadow={opacity=0.7, fill=black!50}]
        \tikzstyle{line} = [draw, -{Stealth[length=2mm, width=1.5mm]}, thick, color=accentcolor]

        % Place nodes
        \node [block] (init) {Initialise};
        \node [block, right=of init] (process) {Process Data};
        \node [block, right=of process] (output) {Output Results};

        % Draw edges
        \path [line] (init) -- (process);
        \path [line] (process) -- (output);
        \path [line, bend left=45] (output) edge node[above, midway, font=\tiny] {Feedback} (init);
    \end{tikzpicture}
    \caption{A styled diagram created with TikZ, showing a simple workflow.}
    \label{fig:tikzflow}
\end{figure}

% Section: Citations
\section{Citations and References}
The template includes support for citations and references using BibLaTeX.

You can cite references from the bibliography, such as \cite{Smith2020} or \cite{Lamport1994}.

\begin{notetbox}
    To use citations, ensure you have the \inlinecode{biber} package installed and run the full compilation process with:
    \begin{lstlisting}[language=bash]
./compile.sh
    \end{lstlisting}
    Or manually:
    \begin{lstlisting}[language=bash]
pdflatex elegant-notebook
biber elegant-notebook
pdflatex elegant-notebook
pdflatex elegant-notebook
    \end{lstlisting}
\end{notetbox}

\subsection{Adding References}
Edit the \inlinecode{references.bib} file to add your own references. The template uses the author-year citation style by default.

\begin{lstlisting}[language=TeX, caption={Example reference entry}, label={lst:reference}]
@article{Smith2020,
  author  = {Smith, John and Johnson, Sarah},
  title   = {Recent Advances in Document Preparation Systems},
  journal = {Journal of Documentation},
  year    = {2020},
  volume  = {76},
  number  = {3},
  pages   = {710--725},
  doi     = {10.1000/example.doi}
}
\end{lstlisting}

% Section: Customisation
\section{Customisation}
The template is designed to be highly customisable.

\subsection{Creating Custom Box Types}
You can create your own custom box types in \inlinecode{config/settings.tex}:

\begin{lstlisting}[language=TeX, caption={Creating a custom box type}, label={lst:custombox}]
\newtcolorbox{warningbox}{
    enhanced,
    colback=red!10,
    colframe=red!80!black,
    fonttitle=\sffamily\bfseries,
    coltitle=white,
    title=Warning,
    left=2mm, right=2mm, top=2mm, bottom=2mm,
    arc=2mm,
    drop shadow,
    breakable
}
\end{lstlisting}

\subsection{Metadata Customisation}
Edit \inlinecode{config/metadata.tex} to personalise your document:
\begin{itemize}
    \item Document title and subtitle
    \item Author information
    \item Institution
    \item Email
    \item Date
\end{itemize}

\subsection{Project Structure}
The template has a well-organised file structure:

\begin{lstlisting}[language=bash, caption={Project directory structure}, label={lst:structure}]
elegant-notebook.tex    # Main document (this file)
references.bib          # Bibliography file
README.md               # Documentation
compile.sh              # Full compilation script
simple-compile.sh       # Basic compilation script
LICENSE                 # MIT Licence
config/                 # Configuration files
    colours.tex         # Colour themes
    metadata.tex        # Document metadata
    settings.tex        # Package imports and settings
    title-page.tex      # Title page layout
\end{lstlisting}

% Section: Best Practices
\section{Best Practices}
Here are some recommendations for using this template effectively:

\begin{itemize}
    \item Keep your document structure organised with clear section hierarchy
    \item Use the provided environments consistently for better visual coherence
    \item Choose a colour and font theme that matches the tone of your document
    \item Compile with \inlinecode{./compile.sh} for the best results, especially when using references
    \item For collaborative work, consider using version control for your LaTeX files
\end{itemize}

\begin{notetbox}
    Remember that consistency is key to professional-looking documents. Stick to the same style choices throughout your document.
\end{notetbox}

% Section: Discussion
\section{Conclusion}
This template provides a flexible, professional-looking document format for academic and research work. Key benefits include:

\begin{itemize}
    \item \textbf{Consistency}: Uniform styling throughout your document
    \item \textbf{Flexibility}: Easy customisation options
    \item \textbf{Professional look}: Clean, modern design with enhanced readability
    \item \textbf{Structured environments}: Ready-to-use boxes for notes, examples, and code
    \item \textbf{British English}: Consistent use of British spelling and conventions
\end{itemize}

We hope this template helps you create beautiful, professional documents for your academic and research needs.

% Section: References
\section{References}
\printbibliography[heading=none]

\end{document} 